\section{Introduction}
\label{introduction}
This course will mainly be concered with the study of topological spaces. Topological spaces are abstract mathematical concepts whose definition include a sufficient amount of data for them to be called ``spaces''. Familiar spaces would be something like $\mathbb{R}^3$ or $\mathbb{R}^n$, but more abstract objects -- such as general vector spaces -- we also think of as ``spaces''. On the other hand, in algebra one encounters objects such as groups and rings that one would generally not think of as spaces, and on the extreme side we talk about sets, which are more general ``collections of things'' which we may or may not choose to think of as spaces.

From this point of view, one might think of topological spaces with an added bit of \word{structure}{struktur}\footnote{See \url{https://en.wikipedia.org/wiki/Mathematical_structure} for a more precise discussion.}; a term used throughout mathematics but typically with a rather vague meaning. As such, sets have no interesting structure, but Euclidian space $\mathbb{R}^n$ has plenty: for instance, the usual inner product $\langle \cdot ,\cdot \rangle$ on $\mathbb{R}^n$ can be used to talk about angles between vectors. This in turn can be used to define the standard norm $\lVert \cdot \rVert$ on $\mathbb{R}^n$ which allows us to talk about lengths of vectors and distances between points. All of this is structure that may not be given to us in a general vector space, but without having this structure on $\mathbb{R}^n$, there would be no such thing as calculus: we wouldn't be able to define things like differentiability and continuity.

For topological spaces we discard all of this fine structure, so that in particular it makes no sense to talk about the distance between two points in a general topological space. The only piece of structure that we will require is that of ``open sets'': given a subset of a topological space, we want to be able to tell if it is open or not. This turns out to be the least amount of structure needed to define continuity, so the study of topological spaces is very much the study of continuous functions.

The study of general topological spaces and continuous functions will be contained in Sections~\ref{topological-spaces}--\ref{homeomorphisms}. Simply having open sets turns out also to be sufficient to talk about what it means for a space to be ``connected'' and ``compact'' in a way that corresponds to what one would normally associate with those words. More abstractly, we will also look at the notion of separating points, which is less familiar in examples like $\mathbb{R}^n$. These properties of topological spaces will be the basis of Sections~\ref{connectedness}--\ref{separation}.

The study of general topological spaces and their fundamental properties is often referred to as \word{point-set topology}{punktm{\"a}ngdstopologi} or \word{general topology}{allm{\"a}n topologi}. The less structure a certain space has, the less deep the mathematical results about it tends to be, and out treatment will involve correspondly few deep mathematical theorems; rather, for the first part of the course, one should think of the materials as developing the necessary tools to deal with topological spaces in other contexts. Towards the end of the course, we will remedy this by tying together our theory with other parts of mathematics. Concretely, in Section~\ref{homotopy} we touch upon the mathematical area of \word{algebraic topology}{algebraisk topologi} which is concerned with analyzing certain natural algebraic structures that can be associated with topological spaces, and in Section~\ref{manifolds}, we will study a particular nice family of topological spaces called manifolds that show up in all of geometry.

Before being able to do any of this, though, we need to firmly settle on what sets are, and how one deals with them.
