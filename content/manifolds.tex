\section{Manifolds}
\label{manifolds}
The concept of a manifold is central in all of differential geometry and mathematical physics; roughly, a manifold is a topological space which locally looks like $\bbR^n$. Another way of viewing it is that a manifold is something which is obtained by gluing together copies of $\bbR^n$. As such, its usefulness in for instance geometry comes from the fact that we can transfer everything we know about calculus on $\bbR^n$ to this much more general family of topological spaces, as long as one ensures that the gluing is sufficiently compatible with calculus. Now, we will not be discussing calculus here but rather take a look at manifolds from a purely topological point of view.

\subsection{Topological manifolds}
\begin{defn}
  \label{definition-locally-euclidean}
  A topological space $X$ is called \word{locally Euclidean}{?} if there is an $n \in \bbN$ so that every point in $X$ has a neighbourhood which is homeomorphic to $\bbR^n$.
\end{defn}
\trans{locally Euclidean}{?}
\begin{defn}
  An \emph{$n$-dimensional} \word{manifold}{m{\aa}ngfald} or simply an \emph{$n$-manifold} is a locally Euclidean second-countable Hausdorff space. The $n$ refers to the $n$ of Definition~\ref{definition-locally-euclidean}.
\end{defn}
\trans{manifold}{m{\aa}ngfald}
Really what we have defined above is a \emph{topological manifold}\index{topological manifold}; since this is the only kind of manifold we will encounter, we will simply call them ``manifolds''.
\begin{example}
  Euclidean space $\bbR^n$ is an $n$-manifold since $\bbR^n$ itself is a neighbourhood of all of its points.
\end{example}
\begin{example}
  \label{spheres-are-manifolds}
  The $n$-sphere $S^n$ is an $n$-manifold. We know by now that a subspace of a Hausdorff space is Hausdorff, and it is not difficult to see that a subspace of a second-countable space is itself second-countable, so we only need to see that $S^n$ is locally Euclidean.
  
  If $x \in S^n$ is a point different from the north pole $p = (0,\dots,0,1)$, then $S^n \setminus \{p\}$ is a neighbourhood of $x$ which is homeomorphic to $\bbR^n$ by Proposition~\ref{north-pole-removed}. If $x = p$, let $q$ denote the south pole. Then $S^n \setminus \{q\}$ is a neighbourhood of $x$ which is homeomorphic to $\bbR^n$ by Remark~\ref{south-pole-removed}.
\end{example}
\begin{lem}
  \label{products-of-manifolds-lemma}
  The product of an $n$-manifold and an $m$-manifold is an $(n+m)$-manifold.
\end{lem}
\begin{proof}
  Exercise~\ref{products-of-manifolds-exercise}.
\end{proof}
\begin{example}
  The $n$-torus $T^n$ is an $n$-manifold by Lemma~\ref{products-of-manifolds-lemma} and Example~\ref{spheres-are-manifolds}.
\end{example}
\begin{example}
  The genus $g$ surfaces $\Sigma_g$ from Example~\ref{surface-example} are $2$-manifolds. Our definition of $\Sigma_g$ is unprecise enough that this is slightly painful to prove; it should, however, be a very reasonable claim, given Figures~\ref{genus-1-surface}-\ref{genus-3-surface}.
\end{example}

\subsection{Embeddings of manifolds}
Notice that by definition, $S^n$ can be embedded in $\bbR^{n+1}$. Similarly, $T^n$ can be embedded in $\bbR^{2n}$, and Figures~\ref{genus-1-surface}-\ref{genus-3-surface} suggest that $\Sigma_g$ can be embedded in $\bbR^3$.

In this section we will see how to use Urysohn's lemma to show the following result.
\begin{thm}
  \label{embedding-of-manifolds}
  Any compact $m$-manifold can be embedded in $\bbR^N$ for some $N \in \bbN$.
\end{thm}
\begin{defn}
  Let $X$ be a topological space and $f : X \to \bbR$ a function. The \word{support}{?} of $f$ is the set
  \[
    \supp(f) = \bar{\{x \mid f(x) \not= 0\}}.
  \]
\end{defn}
\trans{support}{?}
\begin{defn}
  Let $X$ be a topological space, and let $\{U_1, \dots, U_n\}$ be an open cover of $X$. A family $\{\phi_1,\dots,\phi_n\}$ of continuous functions $\phi_i : X \to [0,1]$ is called a \word{partition of unity}{?} dominated by $\{U_i\}$ if
  \begin{itemize}
    \item $\supp(\phi_i) \subset U_i$ for $i = 1, \dots, n$, and
    \item $\sum_{i=1}^n \phi_i(x) = 1$ for all $x \in X$.
  \end{itemize}
\end{defn}
\trans{partition of unity}{?}
\begin{thm}
  \label{t4-partition-of-unity}
  Let $X$ be a $T_4$-space, and let $\{U_1, \dots, U_n\}$ be a finite open cover. Then there exists a partition of unity dominated by $\{U_1,\dots,U_n\}$.
\end{thm}
\begin{proof}
  We first show that we can find an open cover $\{V_1, \dots, V_n\}$ so that $\bar{V_i} \subset U_i$ for all $i$. Consider the set $A_1 = X \setminus (U_2 \cup \dots \cup U_n)$. This is clearly closed, and $A_1 \subset U_1$ since $\{U_i\}$ is a cover. Since $X$ is $T_4$, by Theorem~\ref{separation-squeeze-lemma} we obtain an open set $V_1$ so that $A_1 \subset V_1 \subset \bar{V_1} \subset U_1$, and in particular $\{V_1,U_2,\dots,U_n\}$ is still an open cover. We proceed now by finite induction: suppose that $\{V_1, \dots, V_{k-1},U_k,\dots,U_n\}$ covers $X$, and let
  \[
    A_k = X \setminus (V_1 \cup \dots \cup V_{k-1} \cup U_{k+1} \cup \dots \cup U_n).
  \]
  Then $A_k \subset U_k$, and we find as above an open set $V_k$ with $A_k \subset V_k \subset \bar{V_k} \subset U_k$ so that $\{V_1, \dots, V_k, U_{k+1}, \dots, U_n\}$.
  
  Now go through the same procedure again to obtain an open cover $\{W_1, \dots, W_n\}$ with $\bar{W_i} \subset V_i$ for all $i$. Applying Urysohn's lemma for each $i = 1, \dots, n$, we find continuous functions $\psi_i : X \to [0,1]$ so that $f(X \setminus V_i) = \{0\}$ and $f(\bar{W_i}) = \{1\}$. It follows that
  \[
    \supp(\psi_i) \subset \bar{V_i} \subset U_i.
  \]
  Since $\{W_i\}$ is a cover of $X$, it follows that $\psi(x) = \sum_{i=1}^n \psi_i(x) > 0$ for all $x$. Now define $\phi_i : X \to [0,1]$ by
  \[
    \phi_i(x) = \frac{\psi_i(x)}{\psi(x)}.
  \]
  We then have $\supp(\phi_i) = \supp(\psi_i) \subset U_i$, and for every $x \in X$, we have
  \[
    \sum_{i=1}^n \phi_i(x) = \frac{1}{\psi(x)} \sum_{i=1}^n \psi_i(x) = 1,
  \]
  so $\{\phi_i\}$ is a partition of unity dominated by $\{U_1,\dots,U_n\}$.
\end{proof}
\begin{proof}[Proof of Theorem~\ref{embedding-of-manifolds}]
  Let $X$ be a compact $m$-manifold, and choose for every $x \in X$, choose a neighbourhood $U_x$ of $x$ so that $U_x \simeq \bbR^m$. These will cover $X$, so since $X$ is compact, we obtain a finite open cover $\{U_1, \dots, U_n\}$ together with homeomorphisms $g_i : U_i \to \bbR^m$ for every $i$. Since $X$ is compact and Hausdorff, $X$ is $T_4$ by Proposition~\ref{compact-hausdorff-normal}, so by Theorem~\ref{t4-partition-of-unity} we can find a partition of unity $\{\phi_i\}$ dominated by $\{U_i\}$. Let $A_i = \supp(\phi_i)$, note that $X = U_i \cup (X \setminus A_i)$, and define for each $i = 1, \dots, n$ a function $h_i : X \to \bbR^m$ by
  \[
    h_i(x) = \begin{cases} \phi_i(x)g_i(x), & \text{for $x \in U_i$} \\ 0, & \text{for $x \in X \setminus A_i$.} \end{cases}
  \]
  Notice that $h_i$ is well-defined since $\phi_i(x)g_i(x) = 0$ for $x \in X \setminus A_i$, and $h_i$ is continuous by Lemma~\ref{pasting-lemma}; here one has to check that $x \mapsto \phi_i(x)g_i(x)$ is continuous on $U_i$ which can be seen by Theorem~\ref{sequential-continuity}. The desired embedding will be the map
  \[
    F : X \to \underbrace{\bbR \times \cdots \times \bbR}_{\text{$n$ factors}} \times \underbrace{\bbR^m \times \cdots \times \bbR^m}_{\text{$n$ factors}} \simeq \bbR^{(m+1)n}
  \]
  given by
  \[
    F(x) = (\phi_1(x),\dots,\phi_n(x),h_1(x),\dots,h_n(x))
  \]
  Now $F$ is continuous since the $\phi_i$ and $h_i$ are, so since $X$ is compact, it follows from Corollary~\ref{cont-compact-Hausdorff} that $F$ is an embedding if we can show that $F$ is injective.
  
  Suppose that $F(x) = F(y)$. Then $\phi_i(x) = \phi_i(y)$ and $h_i(x) = h_i(y)$ for all $i$. Since $\sum_{i=1}^n \phi_i(x) = 1$, there is an $i$ with $\phi_i(x) > 0$, so $\phi_i(y) > 0$ as well, which implies that $x, y \in \supp(\phi_i) \subset U_i$. Now
  \[
    \phi_i(x) g_i(x) = h_i(x) = h_i(y) = \phi_i(y) g_i(y),
  \]
  so we must also have $g_i(x) = g_i(y)$. Since each $g_i$ was a homeomorphism, this implies that $x = y$.
\end{proof}

In fact, it turns out that the condition of Theorem~\ref{embedding-of-manifolds} that the manifold is compact is not necessary. Moreover, one could ask how small it is possible to choose $N$ in the theorem; in the proof we saw that an $m$-manifold $X$ can be embedded in $\bbR^{(m+1)n}$, where $n$ is the cardinality of an open cover of $X$ whose constituent open sets are homeomorphic to $\bbR^m$. As the examples in the beginning of this section illustrate however, we should be able to do better: $S^m$ is an $m$-manifold which can be covered by $2$ such open sets, so the theorem provides us with an embedding $S^m \to \bbR^{2(m+1)}$, but we also know that there is also an embedding $S^m \to \bbR^{m+1}$.

The following result extends Theorem~\ref{embedding-of-manifolds} to the non-compact case and provides an explicit bound on the required dimension. We do not include a proof and refer instead to \cite[\S 50, Exercises~6--7]{Mun}.

\begin{thm}
  \label{general-embedding}
  Any $m$-manifold can be embedded in $\bbR^{2m+1}$.
\end{thm}
Recall that manifolds are assumed to be both Hausdorff and second-countable. Since we argued that manifolds are natural objects in geometry, we should provide some motivation for these requirements. Now as we have seen plenty of times, the property of Hausdorff is necessary to do any kind of calculus -- for instance, without this conditions, one would have manifolds with convergent sequences but no unique limit (compare with Proposition~\ref{limits-unique-hausdorff}).

It is less clear though, why we require manifolds to be second-countable, but it turns out that if we did not add this condition, Theorem~\ref{general-embedding} would be false; a counter-example is the so-called long line -- see \cite[\S 24, Exercise~12]{Mun}. Thus insofar that one considers embedding into Euclidean space to be a sufficient amount of motivation, second-countability is necessary.

\subsection{Paracompactness}
As it turns out, one of the most useful tools for studying manifolds are the partitions of unity that we encountered in the previous section. In the proof of Theorem~\ref{embedding-of-manifolds} -- where they played an essential role -- we saw that these exist for compact manifolds but many interesting manifolds are not compact; second-countability provides us with something almost as good. Here, we will illustrate how, referring to \cite{Mun} for most of the proofs.

\begin{defn}
  Let $X$ be a topological space. A collection $\calU$ of subsets of $X$ is called \word{locally finite}{?} if every point of $X$ has a neighbourhood that intersects only finitely many elements of $\calU$.
\end{defn}
\trans{locally finite}{?}
\begin{defn}
  A topological space $X$ is called \word{paracompact}{?} if every open cover has a locally finite subcover.
\end{defn}
\trans{paracompact}{?}
Notice that a finite cover is always locally finite. Thus in particular, all compact spaces are paracompact. The next result says that for paracompact Hausdorff spaces, we always have a locally finite version of partitions of unity. It turns out that such spaces are normal \cite[Thm.~41.1]{Mun} and so the proof is almost identical to that of Theorem~\ref{t4-partition-of-unity}; see \cite[Thm.~41.7]{Mun}
\begin{thm}
  Let $X$ be a paracompact Hausdorff space, and let $\{U_i\}_{i \in I}$ be an open cover of $X$. Then there exists a partition of unity dominated by $\{U_i\}_{i \in I}$; that is, there exists a family $\{\phi_i\}_{i \in I}$ of continuous functions $\phi_i : X \to [0, 1]$ so that
  \begin{itemize}
    \item[(i)] $\supp(\phi_i) \subset U_i$ for all $i \in I$,
    \item[(ii)] $\{\supp(\phi_i)\}_{i \in I}$ is locally finite, and
    \item[(iii)] $\sum_{i \in I} \phi_i(x) = 1$ for every $x \in I$.
  \end{itemize}
\end{thm}
Notice here that the sum appearing in (iii) makes sense because of the locally finiteness from (ii). We end our discussion by noting that in the context of manifolds, paracompactness and second-countability is almost the same thing.
\begin{thm}
  Let $X$ be a locally Euclidean Hausdorff space. Then $X$ is second-countable if and only if $X$ is paracompact and has countably many connected components.
\end{thm}
