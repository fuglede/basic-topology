\section{Manifolds}
\label{manifolds}
The concept of a manifold is central in all of differential geometry and mathematical physics; roughly, a manifold is a topological space which locally looks like $\bbR^n$. Another way of viewing it is that a manifold is something which is obtained by gluing together copies of $\bbR^n$. As such, its usefulness in for instance geometry comes from the fact that we can transfer everything we know about calculus on $\bbR^n$ to this much more general family of topological spaces, as long as one ensures that the gluing is sufficiently compatible with calculus. Now, we will not be discussing calculus here but rather take a look at manifolds from a purely topological point of view.

\subsection{Topological manifolds}
\begin{defn}
  An \emph{$n$-dimensional} \word{manifold}{m{\aa}ngfald} or simply an \emph{$n$-manifold} is a second countable Hausdorff space such that every point has a neighbourhood which is homeomorphic to $\bbR^n$.
\end{defn}
\trans{manifold}{m{\aa}ngfald}
Really what we have defined above is a \emph{topological manifold}\index{topological manifold}; since this is the only kind of manifold we will encounter, we will simply call them ``manifolds''.
\begin{example}
  Euclidean space $\bbR^n$ is a manifold since $\bbR^n$ itself is a neighbourhood of all of its points.
\end{example}
\begin{example}
  \label{spheres-are-manifolds}
  The $n$-sphere $S^n$ is a manifold. If $x \in S^n$ is a point different from the north pole $p = (0,\dots,0,1)$, then $S^n \setminus \{p\}$ is a neighbourhood of $x$ which is homeomorphic to $\bbR^n$ by Proposition~\ref{north-pole-removed}. If $x = p$, let $q$ denote the south pole. Then $S^n \setminus \{q\}$ is a neighbourhood of $x$ which is homeomorphic to $\bbR^n$ by Remark~\ref{south-pole-removed}.
\end{example}
\begin{lem}
  \label{products-of-manifolds-lemma}
  The product of an $n$-manifold and an $m$-manifold is an $(n+m)$-manifold.
\end{lem}
\begin{proof}
  Exercise~\ref{products-of-manifolds-exercise}.
\end{proof}
\begin{example}
  The $n$-torus $T^n$ is an $n$-manifold by Lemma~\ref{products-of-manifolds-lemma} and Example~\ref{spheres-are-manifolds}.
\end{example}
\begin{example}
  The genus $g$ surfaces $\Sigma_g$ from Example~\ref{surface-example} are $2$-manifolds.
\end{example}

\subsection{Embeddings of manifolds}
Notice that by definition, $S^n$ can be embedded in $\bbR^{n+1}$. Similarly, $T^n$ can be embedded in $\bbR^{2n}$, and Figures~\ref{genus-1-surface}-\ref{genus-3-surface} suggest that $\Sigma_g$ can be embedded in $\bbR^3$.

In this section we will see how to use Urysohn's lemma to show the following result.
\begin{thm}
  Any compact $n$-manifold can be embedded in $\bbR^N$ for some $N \in \bbN$.
\end{thm}
\begin{defn}
  Let $X$ be a topological space and $f : X \to \bbR$ a function. The \word{support}{?} of $f$ is the set
  \[
    \supp(f) = \bar{\{x \mid f(x) \not= 0\}}.
  \]
\end{defn}
\trans{support}{?}
\begin{defn}
  Let $X$ be a topological space, and let $\{U_1, \dots, U_n\}$ be an open cover of $X$. A family $\{\phi_1,\dots,\phi_n\}$ of continuous functions $\phi_i : X \to [0,1]$ is called a \word{partition of unity}{?} dominated by $\{U_i\}$ if
  \begin{itemize}
    \item $\supp(\phi_i) \subset U_i$ for $i = 1, \dots, n$, and
    \item $\sum_{i=1}^n \phi_i(x) = 1$ for all $x \in X$.
  \end{itemize}
\end{defn}
\trans{partition of unity}{?}
\begin{thm}
  Let $X$ be a $T_4$-space, and let $\{U_1, \dots, U_n\}$ be a finite open cover. Then there exists a partition of unity dominated by $\{U_1,\dots,U_n\}$.
\end{thm}
\begin{proof}
  We first show that we can find an open cover $\{V_1, \dots, V_n\}$ so that $\bar{V_i} \subset U_i$ for all $i$. Consider the set $A_1 = X \setminus (U_2 \cup \dots \cup U_n)$. This is clearly closed, and $A_1 \subset U_1$ since $\{U_i\}$ is a cover. Since $X$ is $T_4$, by Theorem~\ref{separation-squeeze-lemma} we obtain an open set $V_1$ so that $A_1 \subset V_1 \subset \bar{V_1} \subset U_1$, and in particular $\{V_1,U_2,\dots,U_n\}$ is still an open cover. We proceed now by finite induction: suppose that $\{V_1, \dots, V_{k-1},U_k,\dots,U_n\}$ covers $X$, and let
  \[
    A_k = X \setminus (V_1 \cup \dots \cup V_{k-1} \cup U_{k+1} \cup \dots \cup U_n).
  \]
  Then $A_k \subset U_k$, and we find as above an open set $V_k$ with $A_k \subset V_k \subset \bar{V_k} \subset U_k$ so that $\{V_1, \dots, V_k, U_{k+1}, \dots, U_n\}$.
  
  Now go through the same procedure again to obtain an open cover $\{W_1, \dots, W_n\}$ with $\bar{W_i} \subset V_i$ for all $i$. Applying Urysohn's lemma for each $i = 1, \dots, n$, we find continuous functions $\psi_i : X \to [0,1]$ so that $f(X \setminus V_i) = \{0\}$ and $f(\bar{W_i}) = \{1\}$. It follows that
  \[
    \supp(\psi_i) \subset \bar{V_i} \subset U_i.
  \]
  Since $\{W_i\}$ is a cover of $X$, it follows that $\psi(x) = \sum_{i=1}^n \psi_i(x) > 0$ for all $x$. Now define $\phi_i : X \to [0,1]$ by
  \[
    \phi_i(x) = \frac{\psi_i(x)}{\psi(x)}.
  \]
  We then have $\supp(\phi_i) = \supp(\psi_i) \subset U_i$, and for every $x \in X$, we have
  \[
    \sum_{i=1}^n \phi_i(x) = \frac{1}{\psi(x)} \sum_{i=1}^n \psi_i(x) = 1,
  \]
  so $\{\phi_i\}$ is a partition of unity dominated by $\{U_1,\dots,U_n\}$.
\end{proof}
