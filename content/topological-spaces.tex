\section{Topological spaces}
\label{topological-spaces}
We now turn to the definition of the objects that will be the most interesting to us: topological spaces.

\subsection{Definitions and first examples}
\trans{topology}{topologi}\trans{topological space}{topologiskt rum}
\begin{defn}
  Let $X$ be a set, and let $\calT \subset \calP(X)$ be a collection of subsets of $X$. Then $\calT$ is called a \word{topology}{topologi} if
  \begin{enumerate}
    \item[(T1)] $\emptyset \in \calT$ and $X \in \calT$,
    \item[(T2)] infinite unions of elements of $\calT$ are once again elements of $\calT$; in symbols, if $U_i \in \calT$ for $i \in I$, then $\bigcup_{i \in I} U_i \in \calT$, and
    \item[(T3)] \emph{finite} intersections of elements of $\calT$ are again elements of $\calT$. That is, if $U_1, \dots, U_n \in \calT$, then $U_1 \cap U_2 \cap \dots \cap U_n \in \calT$.
  \end{enumerate}
  If $\calT$ is a topology on $X$, then the pair $(X,\calT)$ is called a \word{topological space}{topologiskt rum}. A set $U \in \calT$ is called \word{open}{{\"o}ppen}.
\end{defn}
\trans{open}{{\"o}ppen}
Note that we will often say that $X$ is a topological space when we mean that $(X,\calT)$ is a topological space. This can be a bit misleading since as the following example shows, a set $X$ might have many different topologies.
\begin{example}
  \label{two-point-topologies}
  Let $X = \{a,b\}$ be a set containing two elements $a$ and $b$. Then each of the four following subsets of $\calP(X)$ define topologies on $X$:
  \begin{align*}
    \calT_1 &= \{\emptyset,X\},\\
    \calT_2 &= \{\emptyset, \{a\}, X\},\\
    \calT_3 &= \{\emptyset, \{b\}, X\}, \\
    \calT_4 &= \{\emptyset,\{a,b\},X\}.
  \end{align*}
  That is, $X$ is a topological space in at least four different ways. In fact, there are $12$ other ways to pick out subsets of $\calP(X)$ but it turns out that these four are the only ones that are topologies. (Rather abstractly, topologies are themselves elements in $\calP(\calP(X))$, which in this case consists of $2^{2^2} = 16$ elements.)
\end{example}
\trans{trivial topology}{den triviala toplogin}\trans{discrete topology}{den diskreta topologin}
\begin{example}
  In fact, any set $X$ can be given a topology in at least two natural ways:
  \begin{itemize}
    \item Let $\calT = \{ \emptyset, X \} \subset \calP(X)$. Then $\calT$ is a topology, which is refered to as \wordexp{the trivial topology}{den triviala topologin}{trivial topology, the}{den triviala toplogin}.
    \item Let $\calT = \calP(X)$ itself. Then $\calT$ is a topology called \wordexp{the discrete topology}{den diskreta topologin}{discrete topology, the}{den diskreta topologin}.
  \end{itemize}
\end{example}
\trans{coarse}{?}\trans{fine}{?}\trans{closed}{sluten}
\begin{defn}
  Let $X$ be a set, and let $\calT$ and $\calT'$ be topologies on $X$. If $\calT \subset \calT'$ then we say that $\calT$ is \word{coarser}{?} than $\calT'$, and that $\calT'$ is \word{finer}{?} than $\calT$. If $\calT \subsetneq \calT'$, we say that $\calT$ is \word{strictly coarser}{?} than $\calT'$, and that $\calT'$ is \word{strictly finer}{?} than $\calT$. If either $\calT \subset \calT'$ or $\calT' \subset \calT$, we say that $\calT$ and $\calT'$ are \word{comparable}{?}.
\end{defn}
\begin{example}
  In Example~\ref{two-point-topologies}, $\calT_2$ is strictly coarser than $\calT_4$, but $\calT_2$ and $\calT_3$ are not comparable. The trivial topology on a set is always coarser than the discrete topology, since $\{\emptyset,X \} \subset \calP(X)$.
\end{example}

\begin{defn}
  A subset $A \subset X$ of a topological space is called \word{closed}{sluten} if $A^c$ is open.
\end{defn}
\begin{prop}
  In a topological space $X$,
  \begin{enumerate}
    \item[(T1')] $\emptyset$ and $X$ are closed,
    \item[(T2')] if $C_i$ are closed for $i \in I$, then $\bigcap_{i\in I} C_i$ is closed, and
    \item[(T3')] if $C_1, \dots, C_n$ are closed, then $C_1 \cup \dots \cup C_n$ is closed.
  \end{enumerate}
\end{prop}

\begin{rem}
  If one has taken a course on measure theory, the definition of a topological space will look familiar: the $\sigma$-algebras appearing for measurable spaces are defined to have particular properties under union, complement, and closure, not unlike topological spaces, but be aware that the two notions are not the same, even though the level of abstraction required to work with them is. However, one could consider the smallest $\sigma$-algebra such that all open sets are measurable (obtaining the so-called Borel sets) and thus turn any topological space into a measurable space in a natural manner. This idea of using some measurable sets to generate a full $\sigma$-algebra is what we will mimic in the following section.
\end{rem}

\subsection{Basis for a topology}
For a given topological space $(X,\calT)$ it can often be a bit clumsy to describe \emph{all} open sets. What one does instead is to describe a certain collection of sets that one wants in $\calT$, and then includes the sets necessary to obtain a full topology, using the rules (T1)--(T3). This idea is contained in what is called a basis for a topology.
\trans{basis}{bas}
\begin{defn}
  Let $X$ be a set, and let $\calB \subseteq \calP(X)$ be any collection of subsets of $X$. Then $\calB$ is called a \word{basis}{bas} for a topology on $X$ if
  \begin{enumerate}
    \item[(B1)] for each $x \in X$, there is a $B \in \calB$ such that $x \in B$, and
    \item[(B2)] if $x \in B_1 \cap B_2$ for $B_1, B_2 \in \calB$, then there is a $B_3 \in \calB$ such that $x \in B_3 \subset B_1 \cap B_2$.
  \end{enumerate}
\end{defn}
If $\calB$ is a basis, we define $\calT_\calB$, the \word{topology generated by}{topologin genererad av} $\calB$ by declaring that $U \in \calT$ if for every $x \in U$, there is a basis element $B \in \calB$ such that $x \in B \subset U$. At first, the condition (B2) might look a little odd but it plays a very explicit role in the proof of the following lemma.
\begin{lem}
  This collection $\calT_\calB \subset \calP(X)$ is a topology.
\end{lem}
\begin{proof}
  Let us show that $\calT_\calB$ satisfies the properties (T1)--(T3) for a topology.

  Notice first that $\emptyset \in \calT_\calB$: a set is in $\calT_\calB$ if all of its elements satisfy a certain condition, but $\emptyset$ contains no elements at all, so the condition is automatically satisfied for all its elements.
  
  That $X \in \calT_\calB$ is exactly (B1). This shows (T1).
  
  To see (T2), let $U_i \in \calT_\calB$ for $i \in I$ and let $x \in \bigcup_{i \in I} U_i$. Then there exists an $i \in I$ so that $x \in U_i$, and since $U_i \in \calT_\calB$ we get a basis element $B \in \calB$ so that $x \in B \subset U_i \subset \bigcup_{i \in I} U_i$. But this says exactly that $\bigcup_{i \in I} U_i \in \calT_\calB$, so this shows (T2).
  
  Finally, to see (T3), let us first show that $U_1 \cap U_2 \in \calT_\calB$ whenever $U_1, U_2 \in \calT_\calB$. To do this, let $x \in U_1 \cap U_2$. Then $x \in U_1$ and $x \in U_2$, so we get sets $B_1, B_2 \in \calB$ so that $x \in B_1 \subseteq U_1$ and $x \in B_2 \subseteq U_2$. Now, by (B2) we get a set $B_3 \in \calB$ so that $x \in B_3 \subset B_1 \cap B_2$. Now clearly, $B_1 \cap B_2 \subset U_1 \cap U_2$ so that we have $x \in B_3 \subset U_1 \cap U_2$, or, in other words, that $U_1 \cap U_2 \in \calT_\calB$.
  
  Finally, let $U_1, \dots, U_n \in \calT_\calB$. Now (T3) follows by induction: if $U_1 \cap \dots \cap U_{n-1} \in \calT_\calB$, then also $U_1 \cap \dots \cap U_n \in \calT_\calB$ since
  \[
    U_1 \cap \dots \cap U_n = (U_1 \cap \dots \cap U_{n-1}) \cap U_n
  \]
  and we now how to handle intersections of only two sets.
\end{proof}
The following proof shows very clearly why we need the condition (B2) in the definition of a basis. The following result gives what might be an easier way to think about $\calT_\calB$.
\begin{lem}
  \label{genereated-by-unions}
  Let $\calB$ be the basis for a topology on a set $X$. Then $U \in \calT_\calB$ if and only if $U = \bigcup_{i \in I} B_i$ for some sets $B_i \in \calB$. That is, $\calT_\calB$ consists of all unions of elements from $\calB$.
\end{lem}
\begin{proof}
  First of all, notice that the $\emptyset$ is the empty union by convention, so we may assume that $U$ is non-empty.

  There are two things to show. First let $U = \bigcup_{i \in I} B_i$ for $B_i \in \calB$, and let $x \in U$. Then there is an $i \in I$ so that $x \in B_i \subset U$. This shows that $U \in \calT_\calB$.
  
  On the other hand, let $U \in \calT_\calB$, and let us see that $U$ is a union of basis elements $B_i$. For every $x \in U$, choose a basis element $B_x$ so that $x \in B_x \subset U$. This is possible since $U \in \calT_\calB$. We now claim that $U = \bigcup_{x \in U} B_x$ which would complete our proof.
  
  To see this, let $y \in U$ be arbitrary. Then $y \in B_y$ and $B_y \subset \bigcup_{x \in U} B_x$, so $y$ is an element of the union. On the other hand, if $y \in \bigcup_{x \in U} B_x$, then there exists a $z \in U$ so that $y \in B_z$, but by our choices of the basis elements, we have that $B_z \subset U$, so $y \in B_z \subset U$.
\end{proof}
While bases are interesting because they allow us to define topologies with less data that we would normally need, we can also go the other way and define a basis that generates a \emph{given} topology; a general way of doing so is the following:
\begin{lem}
  Let $(X,\calT)$ be a topological space. Let $\calC \subset \calT$ be a collection of open sets on $X$ with the following property: for each set $U \in \calT$ and each $x \in U$ there is a $C \in \calC$ so that $x \in C \subset U$. Then $\calC$ is a basis for $\calT$.
\end{lem}
\begin{proof}
  We first show that $\calC$ is a basis by showing that it satisfies (B1) and (B2). To see (B1), let $x \in X$. Since $X \in \calT$ by (T1) we get a $C \in \calC$ so that $x \in C \subset X$ by assumption, so this in particular shows (B1).
  
  Now let $x \in C_1 \cap C_2$ for $C_1,C_2 \in \calC$. Since the sets $C_1$ and $C_2$ are open by assumption, so is $C_1 \cap C_2$. Therefore we get a $C \in \calC$ so that $x \in C \subset C_1 \cap C_2$, which shows (B2).
  
  We now need to show that the topology $\calT_\calC$ that $\calC$ generates is actually $\calT$. First we show that $\calT \subset \calT_\calC$, so let $U \in \calT$. Then for any $x \in \calT$ we can find a $C \in \calC$ so that $x \in C \subset U$ but this is exactly the condition that $U \in \calT_\calC$. On the other hand, if $U \in \calT_\calC$ we know from Lemma~\ref{genereated-by-unions} that $U$ is a union of elements of $\calC$. Since $\calC \subset \calT$ it follows from (T2), applied to $\calT$, that $U \in \calT$.
\end{proof}

\begin{example}
  \label{example-discrete-topology}
  If $X = \{a,b\}$, then $\calB = \{ \{a\}, \{b\} \}$ is a basis for a topology on $X$. The topology $\calT_\calB$ is exactly the discrete topology, $\calT_\calB = \calP(X)$. More generally, let $X$ be any set, and let $\calB$ consist of those sets that contain only a single element, that is
  \[
    \calB = \{ \{x \} \mid x \in X\}.
  \]
  Then $\calB$ is a basis for a topology, and $\calT_\calB$ is the discrete topology.
\end{example}

So far, we have been dealing with abstract sets and topological spaces, but at the end of the day, we will be interested in particular topologies on concrete spaces, so at this point, let us use the notion of a basis for a topology to show how we can easily describe a topology on $\bbR^n$ that agrees with the one we know from analysis.

For $x \in \bbR^n$ and $r > 0$, let
\[
  B(x,r) = \{ y \in \bbR^n \mid \Abs{x-y} < r \}
\]
be the open ball centered in $x$ with radius $r$.

\begin{prop}
  \label{basis-euclidean}
  The collection
  \[
    \calB = \{ B(x,r) \mid x \in \bbR^n, r > 0 \}
  \]
  is the basis for a topology on $\bbR^n$. The resulting topology $\calT_\calB$ is called the standard topology and its open sets are exactly the open sets that one will have encountered in a course on analysis or calculus.
\end{prop}

This result will follow from the more general Proposition~\ref{metric-basis} below. While the standard topology is the most interesting one to consider, below we introduce certain other topologies on $\bbR$.

The following result allows us to compare the topologies generated by bases if we know how to compare the bases. Its proof is similar in spirit to the proofs above: the spaces in question are so abstract and have so little structure that one is forced to use the few things that one actually knows about the spaces. The details can be found in \cite[Lem.~13.3]{Mun}.

\begin{lem}
  \label{compare-bases}
  Let $\calB$ and $\calB'$ be bases for topologies $\calT$ and $\calT'$ respectively. Then the following are equivalent:
  \begin{itemize}
    \item[(1)] The topology $\calT'$ is finer than $\calT$.
    \item[(2)] For every $x \in X$ and each basis element $B \in \calB$ satisfying $x \in B$, there is a basis element $B' \in \calB'$ so that $x \in B' \subset B$.
  \end{itemize}
\end{lem}

\begin{example}
  We can define a basis for a topology on $\bbR$ by
  \[
    \calB_l = \{ x \in \bbR \mid a \leq x < b \}.
  \]
  The topology $\calT_l$ generated by $\calB_l$ is called the \word{lower limit topology}{?} on $\bbR$, and we write $\bbR_l = (\bbR,\calT_l)$.
\end{example}
\begin{example}
  Let $K = \{ 1/n \mid n \in \bbN \} \subset \bbR$ and let $\calB_K$ consist of all open intervals as well as all sets of the form $(a,b) \setminus K$. Then $\calB_K$ is a basis and the topology $\calT_K$ that it generates is called the \word{$K$-topology}{$K$-topologin} on $\bbR$. We write $\bbR_K = (\bbR,\calT_K)$.
\end{example}
So, at this point we have introduced three different topologies on $\bbR$ and we can now use our results above to compare them.
\begin{lem}
  The topologies $\bbR_l$ and $\bbR_K$ are both strictly finer than the standard topology but are not comparable with each other.
\end{lem}
\begin{proof}
  We first show that the topology on $\bbR_l$ is strictly finer than the standard topology. Let $x \in \bbR$. Let $(a,b)$ be an interval containing $x$ -- that is, one of the basis elements for the standard topology. Then $[x,b) \subset (a,b)$ and it follows from Lemma~\ref{compare-bases} that the topology on $\bbR_l$ is finer than the standard topology. It is strictly finer because $[x,b)$ is open in $\bbR_l$ but not in the standard topology: There is no open interval $B$ so that $x \in B \subset [x,b)$.
  
  Similarly for $\bbR_K$: Let $x \in \bbR$ and let $(a,b)$ contain $x$. Then this interval itself belongs to $\calB_K$ so by Lemma~\ref{compare-bases} we have that the topology on $\bbR_K$ is finer than the standard topology. To see that it is strictly finer, consider the set $U = (-1,1) \setminus K \in \calT_K$. Then $0 \in U$ but there is no open interval $B$ so that $0 \in B \subset U$.
  
  Finally, one can show that $U \in \calT_K$ but $U \notin \calT_l$, and that $[1,2) \in \calT_l$ but $[1,2) \notin \calT_K$.
\end{proof}

\subsection{Metric spaces}
Roughly speaking, metric spaces are spaces where one can always measure distances between two points. This makes them a generalisation of $\bbR^n$ and in this section we will see that they are special cases of topological spaces. That is, that having a notion of distance is sufficient to obtain a notion of open sets.

\trans{metric space}{metriskt rum}\trans{metric}{metrik}\trans{distance}{avst{\aa}nd}
\begin{defn}
  A \word{metric space}{metriskt rum} $(X,d)$ is a set $X$ together with a non-negative function $d : X \times X \to \bbR_{\geq 0}$ satisfying for all $x,y,z \in X$ that
  \begin{enumerate}
    \item[(M1)] $d(x,y) = 0$ if and only if $x = y$,
    \item[(M2)] $d(x,y) = d(y,x)$, and
    \item[(M3)] the triangle inequality $d(x,y) + d(y,z) \leq d(x,z)$.
  \end{enumerate}
  The function $d$ is called a \word{metric}{metrik}, and $d(x,y)$ is called the \word{distance}{avst{\aa}nd} from $x$ to $y$.
\end{defn}
Having a metric is sufficient to mimic the definition of open balls that we know for $\bbR^n$. More precisely, for a metric space $(X,d)$ the open \word{ball}{boll} $B_d(x,r)$ centered at $x$, with radius $r > 0$, with respect to the metric $d$ is defined as
\[
  B_d(x,r) = \{ y \in X \mid d(x,y) < r \}.
\]
We will now show how to use the open balls to define a topology, called \word{the metric topology}{den metriska topologin}{metric topology, the}{den metriska topologin}, on any metric space. As promised, this includes Proposition~\ref{basis-euclidean} as a special case.

\begin{prop}
  \label{metric-basis}
  If $(X,d)$ is a metric space, then the collection
  \[
    \calB = \{ B_d(x,r) \mid x \in X, r > 0\}
  \]
  is a basis for a topology.
\end{prop}
\begin{proof}
  We need to show that $\calB$ satisfies (B1) and (B2). Firstly, (B1) follows since $x \in B_d(x,r)$ for any $r > 0$.
  
  To see (B2), let $x \in B_d(y_1,r_1) \cap B_d(y_2,r_2)$ and let us show that there is a $r > 0$ so that
  \begin{align}
    \label{balls-triangle}
    B_d(x,r) \subset B_d(y_1,r_1) \cap B_d(y_2,r_2)
  \end{align}
  Drawing the situation in $\bbR^2$ one sees that the existence of this $r$ is rather reasonable, and that a good guess would be
  \[
    r = \min(r_1 - d(x,y_1) , r_2 - d(x,y_2)),
  \]
  so let us check that this \eqref{balls-triangle} holds with this choice of $r$. Let $z \in B_d(x,r)$ and let us show that $z \in B_d(y_1,r_1)$ and $z \in B_d(y_2,r_2)$. This follows from (M3) as
  \[
    d(z,y_i) \leq d(z,x) + d(x,y_i) < r + d(x,y_i) < r_i
  \]
  for $i = 1,2$.
\end{proof}
\begin{rem}
  One can show from the definition of the induced topology, that a set $U$ is open in the metric topology exactly if for every point $x \in U$ there is an $r > 0$ so that $B_d(x,r) \subset U$, that is, for the case of $\bbR^n$, we recover the usual condition for a set to be open.
\end{rem}
\begin{example}
  As already alluded to above, Euclidean space $\bbR^n$ is a metric space with metric $d(x,y) = \Abs{x-y}$.
\end{example}
\begin{example}
  Let $X$ be any set. Then we can define a metric on $X$ by
  \[
    d(x,y) = \begin{cases} 0, & \text{if $x = y$,} \\ 1, & \text{if $x \not= y$.} \end{cases}
  \]
  The topology induced by this metric is the discrete topology. This follows almost directly from Example~\ref{example-discrete-topology}; let us describe the collection of open balls. Let $x \in X$ be arbitrary. If $r \leq 1$, then $B_d(x,r) = \{x\}$ while if $r > 1$ then $B_d(x,r) = X$. Thus the basis of open balls is
  \[
    \calB = \{ \{x \} \mid x \in X \} \cup \{X\}.
  \]
  Now clearly, every set $U$ in $X$ is a union of sets from this collection since $U = \bigcup_{x \in U} \{x\}$, so it follows that the induced topology consists of all subsets of $X$.
\end{example}


\section{Continuous functions}
As mentioned in the introduction, having the data of open sets turns out to be sufficient to define continuous functions. Recall that if $f : X \to Y$ is a function between two sets, and $A \subset Y$ is a subset, then we define the \word{preimage}{urbild} of $A$ to be the set
\[
  f^{-1}(A) = \{ x \in X \mid f(x) \in A \}.
\]
Be aware that the notation $f^{-1}$ is often used for the inverse of an invertible function, but one does \emph{not} need a function to be invertible to talk about preimages.

\begin{prop}
  The preimage behaves nicely with respect to various operations of sets. In particular, if $f : X \to Y$ and $\{B_i\}_{i \in I}$ is a family of subsets of $Y$, then
  \begin{align*}
    f^{-1}\left( \bigcup_{i \in I} B_i \right) = \bigcup_{i \in I} f^{-1}(B_i), \quad f^{-1}\left( \bigcap_{i \in I} B_i \right) = \bigcap_{i \in I} f^{-1}(B_i).
  \end{align*}
  If $B \subset Y$, then $f^{-1}(B^c) = f^{-1}(B)^c$, and if $g : Y \to Z$ is another map, then
  \[
    (g \circ f)^{-1}(U) = f^{-1}(g^{-1}(U)).
  \]
\end{prop}
\begin{defn}
  Let $(X,\calT_X)$ and $(Y,\calT_Y)$ be topological spaces. A function $f : X \to Y$ is called \word{continuous}{kontinuerlig} if $f^{-1}(U) \in \calT_X$ for all $U \in \calT_Y$, or in words, if the preimages of open sets are open.
  
  A function $f : X \to Y$ is called \emph{continuous at a point $x \in X$} if for every $U \in \calT_Y$ with $f(x) \in U$ there is a $V \in \calT_X$ so that $x \in V$ and $F(V) \subset U$.
\end{defn}
\trans{continuous}{kontinuerlig}
\begin{example}
  Let $X$ be a topological space. Then the identity map $\id : X \to X$ is continuous since $f^{-1}(U) = U$ for every subset $U \subset X$.
\end{example}
\begin{example}
  Let $(X,\calT_X)$ and $(Y,\calT_Y)$ be topological spaces, and let $y \in Y$. Then the constant map $f : X \to Y$, $f(x) = y$ for all $x$, is continuous. To see this, let $U \in \calT_Y$ and let us consider two cases: if $y \in U$, then $f^{-1}(U) = X$ which is open, and if $y \notin U$, then $f^{-1}(U) = \emptyset$, which is also open.
\end{example}
\begin{example}
  Let $X$ have the discrete topology, and let $Y$ be any topological space. Then any map $f : X \to Y$ is continuous, since $f^{-1}(U) \in \calP(X)$ no matter what $U$ is.
\end{example}
\begin{example}
  Let $X$ be any topological space, and let $Y$ have the trivial topology. Then any map $f : X \to Y$ is continuous since $f^{-1}(\emptyset) = \emptyset$, which is open in $X$, and $f^{-1}(Y) = X$, which is also open.
\end{example}
We will soon have a huge family of examples of functions which are \emph{not} continuous; thus in particular the last two examples show that the notion of ``continuity'' depends heavily on the topologies on the spaces under consideration.
\begin{thm}
  The following properties hold for continuous functions:
  \begin{enumerate}
    \item[(i)] If $f: X \to Y$ and $g : Y \to Z$ are continuous, then so is $g \circ f : X \to Z$.
    \item[(ii)] A function $f : X \to Y$ is continuous if and only if the preimage of any closed set is closed.
    \item[(iii)] A function $f : X \to Y$ is continuous if and only if it is continuous at $x$ for all $x \in X$.
  \end{enumerate}
\end{thm}
\begin{proof}
  To see the first part, let $U \subset Z$ be open in $Z$. Then since $U \in \calT_Z$ and $g$ is continous, $g^{-1}(U) \in \calT_Y$, and since $f$ is continuous, we have $(g \circ f)^{-1}(U) = f^{-1}(g^{-1}(U)) \in \calT_X$.
  
  For the second part, suppose first that $f$ is continuous, and let $C \subset Y$ be closed. Then $C^c$ is open, and $f^{-1}(C)^c = f^{-1}(C^c)$ is open. The other direction is similar.
  
  Finally, suppose that $f$ is continuous, let $x \in X$, and let $U \in \calT_Y$ with $f(x) \in U$. Then $V = f^{-1}(U)$ is open in $X$ and $f(V) = U$, so $f$ is continuous at $x$. Suppose on the other hand that $f$ is continuous at $x$ for all $x \in X$, and let $U \in \calT_Y$. Assume without loss of generality that $f^{-1}(U)$ is non-empty, and let $x \in f^{-1}(U)$. Then there exists $V_x \in \calT_X$ so that $x \in V_x$ and $f(V_x) \subset U$. Now $f^{-1}(U)$ is the union of such a collection of such $V_x$ thus open by (T2) since each $V_x$ is.
\end{proof}
