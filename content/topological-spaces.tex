\section{Topological spaces}
\label{topological-spaces}
We now turn to the definition of the objects that will be the most interesting to us: topological spaces.

\subsection{Definitions and first examples}
\trans{topology}{topologi}\trans{topological space}{topologiskt rum}
\begin{defn}
  Let $X$ be a set, and let $\calT \subset \calP(X)$ be a collection of subsets of $X$. Then $\calT$ is called a \word{topology}{topologi} if
  \begin{enumerate}
    \item[(T1)] $\emptyset \in \calT$ and $X \in \calT$,
    \item[(T2)] infinite unions of elements of $\calT$ are once again elements of $\calT$; in symbols, if $U_i \in \calT$ for $i \in I$, then $\bigcup_{i \in I} U_i \in \calT$, and
    \item[(T3)] \emph{finite} intersections of elements of $\calT$ are again elements of $\calT$. That is, if $U_1, \dots, U_n \in \calT$, then $U_1 \cap U_2 \cap \dots \cap U_n \in \calT$.
  \end{enumerate}
  If $\calT$ is a topology on $X$, then the pair $(X,\calT)$ is called a \word{topological space}{topologiskt rum}. A set $U \in \calT$ is called \word{open}{{\"o}ppen}.
\end{defn}
\trans{open}{{\"o}ppen}
Note that we will often say that $X$ is a topological space when we mean that $(X,\calT)$ is a topological space. This can be a bit misleading since as the following example shows, a set $X$ might have many different topologies.
\begin{example}
  \label{two-point-topologies}
  Let $X = \{a,b\}$ be a set containing two elements $a$ and $b$. Then each of the four following subsets of $\calP(X)$ define topologies on $X$:
  \begin{align*}
    \calT_1 &= \{\emptyset,X\},\\
    \calT_2 &= \{\emptyset, \{a\}, X\},\\
    \calT_3 &= \{\emptyset, \{b\}, X\}, \\
    \calT_4 &= \{\emptyset,\{a,b\},X\}.
  \end{align*}
  That is, $X$ is a topological space in at least four different ways. In fact, there are $12$ other ways to pick out subsets of $\calP(X)$ but it turns out that these four are the only ones that are topologies. (Rather abstractly, topologies are themselves elements in $\calP(\calP(X))$, which in this case consists of $2^{2^2} = 16$ elements.)
\end{example}
\trans{trivial topology}{den triviala toplogin}\trans{discrete topology}{den diskreta topologin}
\begin{example}
  In fact, any set $X$ can be given a topology in at least two natural ways:
  \begin{itemize}
    \item Let $\calT = \{ \emptyset, X \} \subset \calP(X)$. Then $\calT$ is a topology, which is refered to as \wordexp{the trivial topology}{den triviala topologin}{trivial topology, the}{den triviala toplogin}.
    \item Let $\calT = \calP(X)$ itself. Then $\calT$ is a topology called \wordexp{the discrete topology}{den diskreta topologin}{discrete topology, the}{den diskreta topologin}.
  \end{itemize}
\end{example}
\trans{coarse}{?}\trans{fine}{?}\trans{closed}{sluten}
\begin{defn}
  Let $X$ be a set, and let $\calT$ and $\calT'$ be topologies on $X$. If $\calT \subset \calT'$ then we say that $\calT$ is \word{coarser}{?} than $\calT'$, and that $\calT'$ is \word{finer}{?} than $\calT$. If $\calT \subsetneq \calT'$, we say that $\calT$ is \word{strictly coarser}{?} than $\calT'$, and that $\calT'$ is \word{strictly finer}{?} than $\calT$. If either $\calT \subset \calT'$ or $\calT' \subset \calT$, we say that $\calT$ and $\calT'$ are \word{comparable}{?}.
\end{defn}
\begin{example}
  In Example~\ref{two-point-topologies}, $\calT_2$ is strictly coarser than $\calT_4$, but $\calT_2$ and $\calT_3$ are not comparable. The trivial topology on a set is always coarser than the discrete topology, since $\{\emptyset,X \} \subset \calP(X)$.
\end{example}

\begin{defn}
  A subset $A \subset X$ of a topological space is called \word{closed}{sluten} if $A^c$ is open.
\end{defn}
\begin{prop}
  \label{prop-closed}
  In a topological space $X$,
  \begin{enumerate}
    \item[(T1')] $\emptyset$ and $X$ are closed,
    \item[(T2')] if $C_i$ are closed for $i \in I$, then $\bigcap_{i\in I} C_i$ is closed, and
    \item[(T3')] if $C_1, \dots, C_n$ are closed, then $C_1 \cup \dots \cup C_n$ is closed.
  \end{enumerate}
\end{prop}

\begin{rem}
  If one has taken a course on measure theory, the definition of a topological space will look familiar: the $\sigma$-algebras appearing for measurable spaces are defined to have particular properties under union, complement, and closure, not unlike topological spaces, but be aware that the two notions are not the same, even though the level of abstraction required to work with them is. However, one could consider the smallest $\sigma$-algebra such that all open sets are measurable (obtaining the so-called Borel sets) and thus turn any topological space into a measurable space in a natural manner. This idea of using some measurable sets to generate a full $\sigma$-algebra is what we will mimic in the following section.
\end{rem}

\subsection{Basis for a topology}
For a given topological space $(X,\calT)$ it can often be a bit clumsy to describe \emph{all} open sets. What one does instead is to describe a certain collection of sets that one wants in $\calT$, and then includes the sets necessary to obtain a full topology, using the rules (T1)--(T3). This idea is contained in what is called a basis for a topology.
\trans{basis}{bas}
\begin{defn}
  Let $X$ be a set, and let $\calB \subseteq \calP(X)$ be any collection of subsets of $X$. Then $\calB$ is called a \word{basis}{bas} for a topology on $X$ if
  \begin{enumerate}
    \item[(B1)] for each $x \in X$, there is a $B \in \calB$ such that $x \in B$, and
    \item[(B2)] if $x \in B_1 \cap B_2$ for $B_1, B_2 \in \calB$, then there is a $B_3 \in \calB$ such that $x \in B_3 \subset B_1 \cap B_2$.
  \end{enumerate}
\end{defn}
If $\calB$ is a basis, we define $\calT_\calB$, the \word{topology generated by}{topologin genererad av} $\calB$ by declaring that $U \in \calT$ if for every $x \in U$, there is a basis element $B \in \calB$ such that $x \in B \subset U$. At first, the condition (B2) might look a little odd but it plays a very explicit role in the proof of the following lemma.
\begin{lem}
  This collection $\calT_\calB \subset \calP(X)$ is a topology.
\end{lem}
\begin{proof}
  Let us show that $\calT_\calB$ satisfies the properties (T1)--(T3) for a topology.

  Notice first that $\emptyset \in \calT_\calB$: a set is in $\calT_\calB$ if all of its elements satisfy a certain condition, but $\emptyset$ contains no elements at all, so the condition is automatically satisfied for all its elements.
  
  That $X \in \calT_\calB$ is exactly (B1). This shows (T1).
  
  To see (T2), let $U_i \in \calT_\calB$ for $i \in I$ and let $x \in \bigcup_{i \in I} U_i$. Then there exists an $i \in I$ so that $x \in U_i$, and since $U_i \in \calT_\calB$ we get a basis element $B \in \calB$ so that $x \in B \subset U_i \subset \bigcup_{i \in I} U_i$. But this says exactly that $\bigcup_{i \in I} U_i \in \calT_\calB$, so this shows (T2).
  
  Finally, to see (T3), let us first show that $U_1 \cap U_2 \in \calT_\calB$ whenever $U_1, U_2 \in \calT_\calB$. To do this, let $x \in U_1 \cap U_2$. Then $x \in U_1$ and $x \in U_2$, so we get sets $B_1, B_2 \in \calB$ so that $x \in B_1 \subseteq U_1$ and $x \in B_2 \subseteq U_2$. Now, by (B2) we get a set $B_3 \in \calB$ so that $x \in B_3 \subset B_1 \cap B_2$. Now clearly, $B_1 \cap B_2 \subset U_1 \cap U_2$ so that we have $x \in B_3 \subset U_1 \cap U_2$, or, in other words, that $U_1 \cap U_2 \in \calT_\calB$.
  
  Finally, let $U_1, \dots, U_n \in \calT_\calB$. Now (T3) follows by induction: if $U_1 \cap \dots \cap U_{n-1} \in \calT_\calB$, then also $U_1 \cap \dots \cap U_n \in \calT_\calB$ since
  \[
    U_1 \cap \dots \cap U_n = (U_1 \cap \dots \cap U_{n-1}) \cap U_n
  \]
  and we now how to handle intersections of only two sets.
\end{proof}
The following proof shows very clearly why we need the condition (B2) in the definition of a basis. The following result gives what might be an easier way to think about $\calT_\calB$.
\begin{lem}
  \label{genereated-by-unions}
  Let $\calB$ be the basis for a topology on a set $X$. Then $U \in \calT_\calB$ if and only if $U = \bigcup_{i \in I} B_i$ for some sets $B_i \in \calB$. That is, $\calT_\calB$ consists of all unions of elements from $\calB$.
\end{lem}
\begin{proof}
  First of all, notice that the $\emptyset$ is the empty union by convention, so we may assume that $U$ is non-empty.

  There are two things to show. First let $U = \bigcup_{i \in I} B_i$ for $B_i \in \calB$, and let $x \in U$. Then there is an $i \in I$ so that $x \in B_i \subset U$. This shows that $U \in \calT_\calB$.
  
  On the other hand, let $U \in \calT_\calB$, and let us see that $U$ is a union of basis elements $B_i$. For every $x \in U$, choose a basis element $B_x$ so that $x \in B_x \subset U$. This is possible since $U \in \calT_\calB$. We now claim that $U = \bigcup_{x \in U} B_x$ which would complete our proof.
  
  To see this, let $y \in U$ be arbitrary. Then $y \in B_y$ and $B_y \subset \bigcup_{x \in U} B_x$, so $y$ is an element of the union. On the other hand, if $y \in \bigcup_{x \in U} B_x$, then there exists a $z \in U$ so that $y \in B_z$, but by our choices of the basis elements, we have that $B_z \subset U$, so $y \in B_z \subset U$.
\end{proof}
While bases are interesting because they allow us to define topologies with less data that we would normally need, we can also go the other way and define a basis that generates a \emph{given} topology; a general way of doing so is the following:
\begin{lem}
  Let $(X,\calT)$ be a topological space. Let $\calC \subset \calT$ be a collection of open sets on $X$ with the following property: for each set $U \in \calT$ and each $x \in U$ there is a $C \in \calC$ so that $x \in C \subset U$. Then $\calC$ is a basis for $\calT$.
\end{lem}
\begin{proof}
  We first show that $\calC$ is a basis by showing that it satisfies (B1) and (B2). To see (B1), let $x \in X$. Since $X \in \calT$ by (T1) we get a $C \in \calC$ so that $x \in C \subset X$ by assumption, so this in particular shows (B1).
  
  Now let $x \in C_1 \cap C_2$ for $C_1,C_2 \in \calC$. Since the sets $C_1$ and $C_2$ are open by assumption, so is $C_1 \cap C_2$. Therefore we get a $C \in \calC$ so that $x \in C \subset C_1 \cap C_2$, which shows (B2).
  
  We now need to show that the topology $\calT_\calC$ that $\calC$ generates is actually $\calT$. First we show that $\calT \subset \calT_\calC$, so let $U \in \calT$. Then for any $x \in \calT$ we can find a $C \in \calC$ so that $x \in C \subset U$ but this is exactly the condition that $U \in \calT_\calC$. On the other hand, if $U \in \calT_\calC$ we know from Lemma~\ref{genereated-by-unions} that $U$ is a union of elements of $\calC$. Since $\calC \subset \calT$ it follows from (T2), applied to $\calT$, that $U \in \calT$.
\end{proof}

\begin{example}
  \label{example-discrete-topology}
  If $X = \{a,b\}$, then $\calB = \{ \{a\}, \{b\} \}$ is a basis for a topology on $X$. The topology $\calT_\calB$ is exactly the discrete topology, $\calT_\calB = \calP(X)$. More generally, let $X$ be any set, and let $\calB$ consist of those sets that contain only a single element, that is
  \[
    \calB = \{ \{x \} \mid x \in X\}.
  \]
  Then $\calB$ is a basis for a topology, and $\calT_\calB$ is the discrete topology.
\end{example}

So far, we have been dealing with abstract sets and topological spaces, but at the end of the day, we will be interested in particular topologies on concrete spaces, so at this point, let us use the notion of a basis for a topology to show how we can easily describe a topology on $\bbR^n$ that agrees with the one we know from analysis.

For $x \in \bbR^n$ and $r > 0$, let
\[
  B(x,r) = \{ y \in \bbR^n \mid \Abs{x-y} < r \}
\]
be the open ball centered in $x$ with radius $r$.

\begin{prop}
  \label{basis-euclidean}
  The collection
  \[
    \calB = \{ B(x,r) \mid x \in \bbR^n, r > 0 \}
  \]
  is the basis for a topology on $\bbR^n$. The resulting topology $\calT_\calB$ is called the standard topology and its open sets are exactly the open sets that one will have encountered in a course on analysis or calculus.
\end{prop}

This result will follow from the more general Proposition~\ref{metric-basis} below. While the standard topology is the most interesting one to consider, below we introduce certain other topologies on $\bbR$.

The following result allows us to compare the topologies generated by bases if we know how to compare the bases. Its proof is similar in spirit to the proofs above: the spaces in question are so abstract and have so little structure that one is forced to use the few things that one actually knows about the spaces. The details can be found in \cite[Lem.~13.3]{Mun}.

\begin{lem}
  \label{compare-bases}
  Let $\calB$ and $\calB'$ be bases for topologies $\calT$ and $\calT'$ respectively. Then the following are equivalent:
  \begin{itemize}
    \item[(1)] The topology $\calT'$ is finer than $\calT$.
    \item[(2)] For every $x \in X$ and each basis element $B \in \calB$ satisfying $x \in B$, there is a basis element $B' \in \calB'$ so that $x \in B' \subset B$.
  \end{itemize}
\end{lem}

\begin{example}
  We can define a basis for a topology on $\bbR$ by
  \[
    \calB_l = \{ x \in \bbR \mid a \leq x < b \}.
  \]
  The topology $\calT_l$ generated by $\calB_l$ is called the \word{lower limit topology}{?} on $\bbR$, and we write $\bbR_l = (\bbR,\calT_l)$.
\end{example}
\begin{example}
  Let $K = \{ 1/n \mid n \in \bbN \} \subset \bbR$ and let $\calB_K$ consist of all open intervals as well as all sets of the form $(a,b) \setminus K$. Then $\calB_K$ is a basis and the topology $\calT_K$ that it generates is called the \word{$K$-topology}{$K$-topologin} on $\bbR$. We write $\bbR_K = (\bbR,\calT_K)$.
\end{example}
So, at this point we have introduced three different topologies on $\bbR$ and we can now use our results above to compare them.
\begin{lem}
  The topologies $\bbR_l$ and $\bbR_K$ are both strictly finer than the standard topology but are not comparable with each other.
\end{lem}
\begin{proof}
  We first show that the topology on $\bbR_l$ is strictly finer than the standard topology. Let $x \in \bbR$. Let $(a,b)$ be an interval containing $x$ -- that is, one of the basis elements for the standard topology. Then $[x,b) \subset (a,b)$ and it follows from Lemma~\ref{compare-bases} that the topology on $\bbR_l$ is finer than the standard topology. It is strictly finer because $[x,b)$ is open in $\bbR_l$ but not in the standard topology: There is no open interval $B$ so that $x \in B \subset [x,b)$.
  
  Similarly for $\bbR_K$: Let $x \in \bbR$ and let $(a,b)$ contain $x$. Then this interval itself belongs to $\calB_K$ so by Lemma~\ref{compare-bases} we have that the topology on $\bbR_K$ is finer than the standard topology. To see that it is strictly finer, consider the set $U = (-1,1) \setminus K \in \calT_K$. Then $0 \in U$ but there is no open interval $B$ so that $0 \in B \subset U$.
  
  Finally, one can show that $U \in \calT_K$ but $U \notin \calT_l$, and that $[1,2) \in \calT_l$ but $[1,2) \notin \calT_K$.
\end{proof}

\subsection{Metric spaces}
Roughly speaking, metric spaces are spaces where one can always measure distances between two points. This makes them a generalisation of $\bbR^n$ and in this section we will see that they are special cases of topological spaces. That is, that having a notion of distance is sufficient to obtain a notion of open sets.

\trans{metric space}{metriskt rum}\trans{metric}{metrik}\trans{distance}{avst{\aa}nd}
\begin{defn}
  A \word{metric space}{metriskt rum} $(X,d)$ is a set $X$ together with a non-negative function $d : X \times X \to \bbR_{\geq 0}$ satisfying for all $x,y,z \in X$ that
  \begin{enumerate}
    \item[(M1)] $d(x,y) = 0$ if and only if $x = y$,
    \item[(M2)] $d(x,y) = d(y,x)$, and
    \item[(M3)] the triangle inequality $d(x,y) + d(y,z) \leq d(x,z)$.
  \end{enumerate}
  The function $d$ is called a \word{metric}{metrik}, and $d(x,y)$ is called the \word{distance}{avst{\aa}nd} from $x$ to $y$.
\end{defn}
Having a metric is sufficient to mimic the definition of open balls that we know for $\bbR^n$. More precisely, for a metric space $(X,d)$ the open \word{ball}{boll} $B_d(x,r)$ centered at $x$, with radius $r > 0$, with respect to the metric $d$ is defined as
\[
  B_d(x,r) = \{ y \in X \mid d(x,y) < r \}.
\]
We will now show how to use the open balls to define a topology, called \word{the metric topology}{den metriska topologin}{metric topology, the}{den metriska topologin}, on any metric space. As promised, this includes Proposition~\ref{basis-euclidean} as a special case.

\begin{prop}
  \label{metric-basis}
  If $(X,d)$ is a metric space, then the collection
  \[
    \calB = \{ B_d(x,r) \mid x \in X, r > 0\}
  \]
  is a basis for a topology.
\end{prop}
\begin{proof}
  We need to show that $\calB$ satisfies (B1) and (B2). Firstly, (B1) follows since $x \in B_d(x,r)$ for any $r > 0$.
  
  To see (B2), let $x \in B_d(y_1,r_1) \cap B_d(y_2,r_2)$ and let us show that there is a $r > 0$ so that
  \begin{align}
    \label{balls-triangle}
    B_d(x,r) \subset B_d(y_1,r_1) \cap B_d(y_2,r_2)
  \end{align}
  Drawing the situation in $\bbR^2$ one sees that the existence of this $r$ is rather reasonable, and that a good guess would be
  \[
    r = \min(r_1 - d(x,y_1) , r_2 - d(x,y_2)),
  \]
  so let us check that this \eqref{balls-triangle} holds with this choice of $r$. Let $z \in B_d(x,r)$ and let us show that $z \in B_d(y_1,r_1)$ and $z \in B_d(y_2,r_2)$. This follows from (M3) as
  \[
    d(z,y_i) \leq d(z,x) + d(x,y_i) < r + d(x,y_i) < r_i
  \]
  for $i = 1,2$.
\end{proof}
\begin{rem}
  \label{remark-def-open-balls}
  One can show from the definition of the induced topology, that a set $U$ is open in the metric topology if and only if for every point $x \in U$ there is an $r > 0$ so that $B_d(x,r) \subset U$, that is, for the case of $\bbR^n$, we recover the usual condition for a set to be open.
  
  To see this, suppose that $U$ is open in the metric topology, and let $x \in U$. Since the topology is induced by the basis of open balls, there exists an open ball $B_d(y,\eps)$ so that $x \in B_d(y,\eps) \subset U$. By setting $r = \eps - d(x,y) > 0$ we see that
  \[
    x \in B_d(x,r) \subset B_d(y,\eps) \subset U.
  \]
  Likewise, the other direction follows from the definition of the topology as induced by the basis of open balls.
\end{rem}
\begin{example}
  As already alluded to above, Euclidean space $\bbR^n$ is a metric space with metric $d(x,y) = \Abs{x-y}$.
\end{example}
\begin{example}
  Let $X$ be any set. Then we can define a metric on $X$ by
  \[
    d(x,y) = \begin{cases} 0, & \text{if $x = y$,} \\ 1, & \text{if $x \not= y$.} \end{cases}
  \]
  The topology induced by this metric is the discrete topology. This follows almost directly from Example~\ref{example-discrete-topology}; let us describe the collection of open balls. Let $x \in X$ be arbitrary. If $r \leq 1$, then $B_d(x,r) = \{x\}$ while if $r > 1$ then $B_d(x,r) = X$. Thus the basis of open balls is
  \[
    \calB = \{ \{x \} \mid x \in X \} \cup \{X\}.
  \]
  Now clearly, every set $U$ in $X$ is a union of sets from this collection since $U = \bigcup_{x \in U} \{x\}$, so it follows that the induced topology consists of all subsets of $X$.
\end{example}


\subsection{Continuous functions}
As mentioned in the introduction, having the data of open sets turns out to be sufficient to define continuous functions. Recall that if $f : X \to Y$ is a function between two sets, and $A \subset Y$ is a subset, then we define the \word{preimage}{urbild} of $A$ to be the set
\[
  f^{-1}(A) = \{ x \in X \mid f(x) \in A \}.
\]
Be aware that the notation $f^{-1}$ is often used for the inverse of an invertible function, but one does \emph{not} need a function to be invertible to talk about preimages.

\begin{prop}
  \label{preimage-props}
  The preimage behaves nicely with respect to various operations of sets. In particular, if $f : X \to Y$ and $\{B_i\}_{i \in I}$ is a family of subsets of $Y$, then
  \begin{align*}
    f^{-1}\left( \bigcup_{i \in I} B_i \right) = \bigcup_{i \in I} f^{-1}(B_i), \quad f^{-1}\left( \bigcap_{i \in I} B_i \right) = \bigcap_{i \in I} f^{-1}(B_i).
  \end{align*}
  If $B \subset Y$, then $f^{-1}(B^c) = f^{-1}(B)^c$, and if $g : Y \to Z$ is another map, then
  \[
    (g \circ f)^{-1}(U) = f^{-1}(g^{-1}(U)).
  \]
\end{prop}
\begin{defn}
  \label{continuity-def}
  Let $(X,\calT_X)$ and $(Y,\calT_Y)$ be topological spaces. A function $f : X \to Y$ is called \word{continuous}{kontinuerlig} if $f^{-1}(U) \in \calT_X$ for all $U \in \calT_Y$, or in words, if the preimages of open sets are open.
  
  A function $f : X \to Y$ is called \emph{continuous at a point $x \in X$} if for every $U \in \calT_Y$ with $f(x) \in U$ there is a $V \in \calT_X$ so that $x \in V$ and $F(V) \subset U$.
\end{defn}
\trans{continuous}{kontinuerlig}
\begin{example}
  Let $X$ be a topological space. Then the identity map $\id : X \to X$ is continuous since $f^{-1}(U) = U$ for every subset $U \subset X$.
\end{example}
\begin{example}
  Let $(X,\calT_X)$ and $(Y,\calT_Y)$ be topological spaces, and let $y \in Y$. Then the constant map $f : X \to Y$, $f(x) = y$ for all $x$, is continuous. To see this, let $U \in \calT_Y$ and let us consider two cases: if $y \in U$, then $f^{-1}(U) = X$ which is open, and if $y \notin U$, then $f^{-1}(U) = \emptyset$, which is also open.
\end{example}
\begin{example}
  Let $X$ have the discrete topology, and let $Y$ be any topological space. Then any map $f : X \to Y$ is continuous, since $f^{-1}(U) \in \calP(X)$ no matter what $U$ is.
\end{example}
\begin{example}
  Let $X$ be any topological space, and let $Y$ have the trivial topology. Then any map $f : X \to Y$ is continuous since $f^{-1}(\emptyset) = \emptyset$, which is open in $X$, and $f^{-1}(Y) = X$, which is also open.
\end{example}
We will soon have a huge family of examples of functions which are \emph{not} continuous; thus in particular the last two examples show that the notion of ``continuity'' depends heavily on the topologies on the spaces under consideration.
\begin{thm}
  \label{continuous-props}
  The following properties hold for continuous functions:
  \begin{enumerate}
    \item[(i)] If $f: X \to Y$ and $g : Y \to Z$ are continuous, then so is $g \circ f : X \to Z$.
    \item[(ii)] A function $f : X \to Y$ is continuous if and only if the preimage of any closed set is closed.
    \item[(iii)] A function $f : X \to Y$ is continuous if and only if it is continuous at $x$ for all $x \in X$.
  \end{enumerate}
\end{thm}
\begin{proof}
  To see the first part, let $U \subset Z$ be open in $Z$. Then since $U \in \calT_Z$ and $g$ is continous, $g^{-1}(U) \in \calT_Y$, and since $f$ is continuous, we have $(g \circ f)^{-1}(U) = f^{-1}(g^{-1}(U)) \in \calT_X$.
  
  For the second part, suppose first that $f$ is continuous, and let $C \subset Y$ be closed. Then $C^c$ is open, and $f^{-1}(C)^c = f^{-1}(C^c)$ is open. The other direction is similar.
  
  Finally, suppose that $f$ is continuous, let $x \in X$, and let $U \in \calT_Y$ with $f(x) \in U$. Then $V = f^{-1}(U)$ is open in $X$ and $f(V) = U$, so $f$ is continuous at $x$. Suppose on the other hand that $f$ is continuous at $x$ for all $x \in X$, and let $U \in \calT_Y$. Assume without loss of generality that $f^{-1}(U)$ is non-empty, and let $x \in f^{-1}(U)$. Then there exists $V_x \in \calT_X$ so that $x \in V_x$ and $f(V_x) \subset U$. Now $f^{-1}(U)$ is the union of such a collection of such $V_x$ thus open by (T2) since each $V_x$ is.
\end{proof}

As the reader will likely have encountered the concept of continuity in other contexts, let us now show that these notions actually coincide. For convenience, let us recall what continuity ``normally'' means.

\begin{defn}
  A function $f : \bbR^n \to \bbR$ is called \emph{continuous at a point} $x \in \bbR^n$ if for every $\eps > 0$ there exists is a $\delta > 0$ such that for every $y \in \bbR^n$ with $\Abs{x-y} < \delta$, one has $\abs{f(x)-f(y)} < \eps$. A function is called \emph{continuous} if it is continuous in every point.
\end{defn}
This definition can be mirrored for general metric spaces by replacing the distances induced by the norms by the metrics.
\begin{thm}
  \label{continuity-def-equivalent}
  Let $(X,d_X)$ and $(Y,d_Y)$ be metric spaces with their induced metric topologies. Then a function $f : X \to Y$ is continuous, in the sense of Definition~\ref{continuity-def} if and only if
  \[
    \forall x \in X, \forall \eps > 0, \exists \delta > 0 : d_X(x,y) < \delta \Rightarrow d_Y(f(x),f(y)) < \eps.
  \]
\end{thm}
We will prove this theorem in a second, using the following result.
\begin{lem}
  \label{continuity-def-equivalent-lemma}
  Let $(X,d_X)$, $(Y,d_Y)$ be metric spaces with the metric topologies. Then a function $f : X \to Y$ is continuous at a point $x \in X$, in the sense of Definition~\ref{continuity-def}, if and only if
  \begin{align}
    \label{cont-at-point-balls}
    \forall \eps > 0, \exists \delta > 0 : f(B_{d_X}(x,\delta)) \subset B_{d_Y}(f(x),\eps).
  \end{align}
\end{lem}
\begin{proof}
  Suppose first that $f$ is continuous at a given point $x \in X$ and let $\eps > 0$. Since $f(x)$ lies in the open set $B_{d_Y}(f(x),\eps)$, there is an open set $V$ in $X$ such that $x \in V$ and $f(V) \subset B_{d_Y}(f(x),\eps)$. By the discussion in Remark~\ref{remark-def-open-balls} the openness of $V$ implies that there is a $\delta > 0$ so that $B_{d_X}(x,\delta) \subset V$, and then in particular $f(B_{d_X}(x,\delta)) \subset f(V) \subset B_{d_Y}(f(x),\eps)$.
  
  Suppose now that \eqref{cont-at-point-balls} holds for $f$ and let $U$ be an open set in $Y$ containing $f(x)$. Once again, by Remark~\ref{remark-def-open-balls}, this implies that there exists an $\eps > 0$ so that $B_{d_Y}(f(x),\eps) \subset U$. By \eqref{cont-at-point-balls} we then get a $\delta > 0$ with $f(B_{d_X}(x,\delta)) \subset B_{d_Y}(f(x),\eps)$, and since $B_{d_X}(x,\delta)$ is open in $X$ and contains $x$ we are done.
\end{proof}
\begin{proof}[Proof of Theorem~\ref{continuity-def-equivalent}]
  This follows by combining Lemma~\ref{continuity-def-equivalent-lemma} and the last part of Theorem~\ref{continuous-props}.
\end{proof}


\subsection{The subspace topology}
Often, we will be dealing with subsets of topological spaces, and we want to be able to consider these subsets as topological spaces in their own right.
\trans{subspace topology}{underrumstopologi}\trans{relative topology}{relativ topologi}
\begin{defn}
  Let $(X,\calT)$ be a topological space, and let $Y \subset X$ be any subset of $X$. Then the collection
  \[
    \calT_Y = \{ Y \cap U \mid U \in \calT \}
  \]
  is called the \wordexp{the subspace topology}{underrumstopologin}{subspace topology, the}{underrumstopologi}.
\end{defn}
\begin{lem}
  The collection $\calT_Y$ actually defines a topology on $Y$.
\end{lem}
\begin{proof}
  Obviously $\emptyset$ and $Y$ are in $\calT_Y$, so (T1) holds. Let $\{U_i\}_{i\in I}$ be a family of subsets $U_i \in \calT_Y$. That is, for every $i \in Y$ there exists a $V_i \in \calT$ so that $U_i = Y \cap V_i$. Then by (an infinite version) of De Morgan's law (Proposition~\ref{de-morgan}),
  \[
    \bigcup_{i \in I} U_i = \bigcup_{i \in I} Y \cap V_i = Y \cap \bigcup_{i \in I} V_i,
  \]
  and since $\bigcup_{i \in I} V_i \in \calT$ by (T2), applied to $\calT$, this shows (T2) for $\calT_Y$.
  
  Finally, (T3) follows by the other of De Morgan's laws by the exact same logic.
\end{proof}
Equipping $Y$ with the subspace topology, we will call $Y$ a \emph{subspace}\index{subspace} of $X$. If a set $U$ belongs to $\calT_Y$ we will often say that \emph{$U$ is open in $Y$}.
\begin{example}
  As a word of warning, a subspace might have open sets that would not be considered open in the full topological space. For instance, let $X = \bbR$ and $Y = [0,\infty)$. Then the half-open interval $[0,1)$ is open in $Y$ since $[0,1) = Y \cap (-1,1)$, but $[0,1)$ is not open in $X$.
\end{example}
\begin{prop}
  \label{props-subspace-top}
  Let $(X,\calT)$ be a topological space, and let $(Y,\calT_Y)$ be a subspace. Then
  \begin{enumerate}
    \item[(i)] the inclusion map $\iota : Y \to X$ given by $\iota(y) = y$ is continuous,
    \item[(ii)] if $Z$ is a topological space, and $f : X \to Z$ is a continuous map, then the restriction $f|_Y : Y \to Z$ is also continuous, and
    \item[(iii)] a set $F \subset Y$ is closed in $Y$ if and only if there is a set $G \subset X$ which is closed in $X$ so that $F = Y \cap G$.
  \end{enumerate}
\end{prop}
\begin{proof}
  To see (i), let $U$ be open in $X$. Then $\iota^{-1}(U) = U \cap Y$, which is open in $Y$ by definition, so $\iota$ is continuous.
  
  To see (ii), let $U$ be open in $Z$. Then $f|_Y^{-1}(U) = f^{-1}(U) \cap Y$, which is open in $Y$ since $f^{-1}(U)$ is open in $X$.
  
  Finally, let us show (iii). Let $F \subset Y$ be closed in $Y$. This means that $Y \setminus F$ is open in $Y$ so there is an set $U$ which is open in $X$ and $Y \setminus F = Y \cap U$. Let $G = X \setminus U$. Then $G$ is closed in $X$ and
  \[
    F = Y \setminus (Y \setminus F) = Y \setminus (Y \cap U) = (Y \setminus Y) \cup (Y \setminus U) = Y \setminus U = Y \cap (X \setminus U) = Y \cap G.
  \]
  On the other hand, let $G \subset X$ be closed so that $X \setminus G$ is open, and let $F = Y \cap G$. We have to show that $F$ is closed in $Y$. We know that $Y \cap (X \setminus G)$ is open in $Y$ and find that
  \[
    Y \cap (X \setminus G) = Y \setminus (Y \cap G) = Y \setminus F,
  \]
  so $F$ is closed in $Y$.
\end{proof}
\begin{example}
  The subspace topology on $\bbZ \subset \bbR$ is the discrete topology on $\bbZ$: the set $\{n\}$ is open in $\bbZ$ for any integer $n$.
  
  On the other hand, the subspace on $\bbQ \subset \bbR$ is \emph{not} the discrete topology, essentially because any non-empty open interval in $\bbR$ contains infinitely many rational numbers.
\end{example}

In metric spaces, all subsets are automatically metric spaces as one can restrict metrics to subsets. The following result shows that the subspace topology gives the ``right'' thing in this case.

\begin{prop}
  If $(X,d)$ is a metric space and $Y \subset X$, then the metric topology induced by the restricted metric $d|_{Y \times Y}$ is exactly the subspace topology on $Y$.
\end{prop}
\begin{proof}
  Left to the reader.
\end{proof}
Finally, let us show the following result, which says that to check that a function is continuous, it suffices to check it on a collection of open (or closed) sets that together make up the entire space.
\begin{example}
  For the familiar topological spaces, this should not be a big surprise: if a function $f : \bbR \to \bbR$ is continuous on $(-\infty,1)$ and $(-1,\infty)$, then it is continuous on all of $(-\infty,\infty)$.
\end{example}

\begin{lem}[The pasting lemma]
  Let $X$ and $Y$ be topological spaces, and let $U,V \subset X$ be two open subsets such that $X = U \cup V$. Let $f : U \to Y$ and $g : V \to Y$ be two functions so that $f|_{U \cap V} = g|_{U \cap V}$. Then $f$ and $g$ are continuous with respect to the the subspace topologies on $U$ and $V$ if and only if the function $h : X \to Y$ given by
  \[
    h(x) = \begin{cases} f(x), & \text{if $x \in U$}, \\ g(x), & \text{if $x \in V$,} \end{cases}
  \]
  is continuous.
\end{lem}
\begin{proof}
  Notice first of all that $h$ is well-defined since $f|_{U \cap V} = g|_{U \cap V}$. If $h$ is continuous, then so are $f$ and $g$ by Proposition~\ref{props-subspace-top}.
  
  Assume that $f$ and $g$ are continuous and let $W \subset Y$ be open in $Y$. Choose open sets $U'$ and $V'$ in $X$ so that $f^{-1}(W) = U \cap U'$ and $g^{-1}(W) = V \cap V'$. Now
  \begin{align*}
    h^{-1}(W) &= \{ x \in X \mid h(x) \in W \} = \{x \in U \mid h(x) \in W \} \cup \{x \in V \mid h(x) \in W \} \\
              &= \{ x \in U \mid f(x) \in W \} \cup \{ x \in V \mid g(x) \in W \} \\
              &= (f^{-1}(W) \cap U) \cup (g^{-1}(W) \cap W) = (U \cap U') \cup (V \cup V'),
  \end{align*}
  which is open in $X$ since we assumed that both $U$ and $V$ were.
\end{proof}
\begin{rem}
  Notice that the exact same result would hold if we replaced ``open'' with ``closed'' everywhere in the statement and proof.
  
  Notice also that the result would also be true if we replaced $U$ and $V$ with an infinite collection of open (or closed) sets $\{U_i\}_{i \in I}$ so that $X = \bigcup_{i \in I} U_i$.
\end{rem}

\subsection{The poset and order topologies}
Recall from Definition~\ref{def-poset} the definition of a poset $(X,\preceq)$.
\begin{prop}
  \label{poset-top-def}
  Let $(X,\preceq)$ be a poset and define for every $a \in X$ a subset
  \[
    P_a = \{ x \in X \mid a \preceq x\}.
  \]
  Then the collection
  \[
    \calB = \{ P_a \mid a \in X \}
  \]
  is the basis for a topology $\calT_\calB$ called the \word{poset topology}{pom{\"a}ngdtopologin} on $X$.
\end{prop}
\begin{proof}
  For every $x \in X$ we have that $x \preceq x$ so $x \in P_x$ which implies (B1).
  
  Suppose that $x \in P_a \cap P_b$ for $a,b \in X$. We claim that $x \in P_x \subset P_a \cap P_b$. This holds since we know that $a \preceq x$ and $b \preceq x$ so for any $y$ with $x \preceq y$, transitivity implies that $a \preceq y$ and $b \preceq y$ so $y \in P_a \cap P_b$.
\end{proof}
For any poset $(X,\preceq)$, we can define \emph{intervals}\index{interval}
\begin{align*}
  [a,b] &= \{ x \in X \mid a \preceq x \preceq b \},\\
  [a,b) &= \{ x \in X \mid a \preceq x \preceq b, x \not= b \},\\
  (a,b] &= \{ x \in X \mid a \preceq x \preceq b, x \not= a \},\\
  (a,b) &= \{ x \in X \mid a \preceq x \preceq b, x\not= a, x \not= b \}.
\end{align*}
We say that an element $a_0 \in X$ is the \emph{smallest element}\index{smallest element} of $X$ if $a_0 \preceq x$ for all $x \in X$, and similarly, we say that $b_0$ is the \emph{largest element}\index{largest element} of $X$ if $x \preceq b_0$ for all $x \in X$. Notice that the word \emph{the} is justified since there can be at most one smallest and one largest element. There need not be any smallest/largest elements at all though, as is evident from the example of the poset $(\bbR,\leq)$.
\trans{order topology}{?}
\begin{prop}
  For a poset $(X,\preceq)$ define a collection $\calB$ of subsets to consist of
  \begin{itemize}
    \item all open intervals $(a,b)$, $a,b \in X$,
    \item all intervals $[a_0,b)$, $b \in X$, if $X$ has a smallest element $a_0$, and
    \item all intervals $(a,b_0]$, $a \in X$, if $X$ has a largest element $b_0$.
  \end{itemize}
  Then $\calB$ is the basis for a topological $\calT_\calB$ on $X$, called the \word{order topology}{?}.
\end{prop}
\begin{proof}
  One needs to check (B1) and (B2) for $\calB$. We leave this to the reader.
\end{proof}
\begin{example}
  Since $(\bbR,\leq)$ has no smallest or largest elements, the basis for its order topology simply consists of all open intervals. That is, the order topology is exactly the standard topology on $\bbR$.
\end{example}

\subsection{The product topology}
Recall from Section~\ref{cartesian-product} how to define, for any collection of sets, their Cartesian product. In this section we will show how to define a topology on a product of topological spaces.

\trans{subbasis}{?}
\begin{defn}
  Let $X$ be a set. A \word{subbasis}{?} $\calC$ for a topology on $X$ is a collection of subsets that cover $X$, meaning that their union is all of $X$. The topology $\calT_\calC$ generated by $\calC$ consists of all unions of all finite intersections of $\calC$. It is the coarsest topology containing $\calC$ meaning that is has as few open sets as possible while still including the elements in $\calC$ as open sets.
\end{defn}
Consider now a cartesian product $X = \prod_{i \in I} X_i$ for a family of sets $\{ X_i \}_{i \in I}$. For every $i \in I$, there is a natural map $\pi_i : X \to X_i$, called the projection onto $X_i$, which maps $\pi_i(x) = x(i)$, where here we think of $x \in X$ as a map $I \to \bigcup_{i \in I} X_i$.

\trans{product topology}{produkttopologin}
\begin{defn}
  Let $\{ X_i \}_{i \in I}$ be a family of topological spaces, and let $X = \prod_{i \in I} X_i$. We then define a topology on $X$, called the \word{product topology}{produkttopologin}, to be the coarsest topology such that $\pi_i$ is continuous for every $i$.
\end{defn}
This definition is rather abstract: rather than describing the actual open sets, we have described a property that we would like the topology to have, namely that all the projections be continuous. Spelled out, the topology on the product $X$ is generated by the subbasis $\calC$ which consists of all sets of the form $\pi_i^{-1}(U)$, where $U$ is an open set in $X_i$.

To make this more concrete, let us consider the case of a product of just two topological spaces $(X_1,\calT_{X_1})$ and $(X_2,\calT_{X_2})$, and let $U$ and $V$ be open sets in $X_1$ and $X_2$ respectively. Then $\pi_1^{-1}(U) = U \times X_2$ and $\pi_2^{-1}(V) = X_1 \times V$ are examples of open sets in $X_1 \times X_2$. Their intersection is the set $\pi_1^{-1}(U) \cap \pi_2^{-1}(V) = U \times V$, and the topology on $X_1 \times X_2$ consists of all unions of this form. In symbols, if we let
\[
  \calB = \{ U \times V \mid U \in \calT_{X_1}, V \in \calT_{X_2} \},
\]
then $\calB$ is a basis for the product topology on $X_1 \times X_2$. Similarly, open sets in $X = \prod_{i \in I} X_i$ are unions of sets of the form $\prod_{i \in I} U_i$, where $U_i$ is open in $X_i$ for each $i \in I$.

\begin{thm}
  Let $X$ be a topological space, and let $\{Y_i\}_{i \in I}$ be a family of topological spaces. A function $f : X \to \prod_{i \in I} Y_i$ consists of a family of functions $\{f_i\}_{i \in I}$ where $f_i : X \to Y_i$ for all $i \in I$. Then $f$ is continuous if and only if $f_i$ is continuous for every $i$.
\end{thm}
\begin{proof}
  Notice first of all that the maps $f_i$ are exactly the compositions $\pi_i \circ f$.
  
  Suppose that $f$ is continuous. Since each $\pi_i$ is continuous, so is every $f_i$ by Theorem~\ref{continuous-props}.
  
  Suppose now that all the $f_i$ are continuous, and let us show that the preimages of elements of the subbasis are open. That is, let $U$ be an open set in $\prod_{i \in I} Y_i$ of the form $U = \pi_i^{-1}(V)$ where $V$ is open in $Y_i$. Then $f^{-1}(U) = f^{-1}(\pi_i^{-1}(V)) = f_i^{-1}(V)$, which is open by assumption. A general open set is a union of finite intersections of elements from the subbasis, so the general case follows from Proposition~\ref{preimage-props}.
\end{proof}

% It would make sense to talk about universal properties here. Maybe I'll do that.

\subsection{Interior, closure, boundary, and limit points}
\begin{defn}
  Let $(X,\calT)$ be a topological space, and let $Y \subset X$ be a subset. Then
  \begin{enumerate}
    \item[(i)] the \word{interior}{inre} of $Y$, denoted $\mathring{Y}$ or $\Int Y$, is the union of all open subsets in $Y$,
    \item[(ii)] the \word{closure}{slutna h{\"o}ljet} of $Y$, denoted $\bar{Y}$, is the intersection of all closed subsets containing $Y$.
    \item[(iii)] The subset $Y$ is called \word{dense}{t{\"a}t} in $X$ if $\bar{Y} = X$.
  \end{enumerate}
\end{defn}
\trans{interior}{inre}\trans{closure}{slutet h{\"o}lje}\trans{boundary}{rand}\trans{dense}{t{\"a}t}
Notice that $\Int Y$ is open, since it is a union of open sets. Likewise, $\bar Y$ is closed by Proposition~\ref{prop-closed}, and we have
\[
  \Int Y \subset Y \subset \bar{Y}.
\]
It also follows directly from the definition that $Y$ is open if and only $Y = \Int Y$ and that $Y$ is closed if and only if $Y = \bar Y$. The definition also implies that $\Int Y$ is the largest open subset contained in $Y$, and that $\bar Y$ is the smallest closed subset containing $Y$.

Furthermore, it is useful to note that the complement of an open set contained in $Y$ is a closed set containing $Y^c$ and on the other hand, the complement of a closed set containing $Y$ is an open set contained in $Y^c$. This implies that
\begin{align}
  \label{int-versus-closure}
  \Int Y = X \setminus \bar{(X \setminus Y)}, \quad \bar{Y} = X \setminus \Int(X \setminus Y),
\end{align}
which can also be shown more precisely by using De Morgan's laws.
\begin{prop}
  Let $Y$ and $Z$ be subsets of a topological space $X$. Then
  \begin{enumerate}
    \item[(i)] $\bar{Y \cup Z} = \bar Y \cup \bar Z$,
    \item[(ii)] $\bar{Y \cap Z} \subset \bar Y \cap \bar Z$,
    \item[(iii)] $\Int Y \cup \Int Z \subset \Int(Y \cup Z)$, and
    \item[(iv)] $\Int Y \cap \Int Z = \Int (Y \cap Z)$.
  \end{enumerate}
\end{prop}
\begin{proof}
  Let us show (i) and (ii); (iii) and (iv) follow by the same logic. First, note that since $Y \subset \bar Y$ and $Z \subset \bar Z$ we get that $Y \cup Z \subset \bar Y \cup \bar Z$. This says that $\bar Y \cup \bar Z$ is a closed subset containing $Y \cup Z$; since $\bar{Y \cup Z}$ is the smallest closed subset containing $Y \cup Z$, this tells us that $\bar{Y \cup Z} \subset \bar Y \cup \bar Z$.
  
  On the other hand $Y \subset Y \cup Z \subset \bar {Y \cup Z}$ and the latter set is closed so $\bar Y \subset \bar Y \cup \bar Z$. For the same reason $\bar Z \subset \bar Y \cup \bar Z$, and this implies that $\bar Y \cup \bar Z \subset \bar { Y \cup Z}$.
  
  Since $Y \subset \bar Y$ and $Z \subset \bar Z$ we have $Y \cap Z \subset \bar Y \cap \bar Z$. The latter set is closed so $\bar {Y\cap Z} \subset \bar Y \cap \bar Z$.
\end{proof}
For any topological space $X$, we say that $U$ is a \word{neighbourhood}{omgivning} of a point $x$ if $x \in U$.


\begin{defn}
  Let $Y$ be a subset of a topological space $X$. Then
  \begin{enumerate}
    \item[(i)] the \word{boundary}{rand} of $Y$, denoted $\dd Y$ is the set
      \[
        \dd Y = \{ x \in X \mid \text{$U \cap Y \not= \emptyset$ and $U \cap Y^c \not= \emptyset$ for all neighbourhoods $U$ of $x$}\}.
      \]
      That is, $x \in \dd Y$ if all neighbourhoods of $x$ intersect both $Y$ and $Y^c$.
    \item[(ii)] A \word{limit point}{gr{\"a}nspunkt} of $Y$ is a point $x \in X$ with the property that all its neighbourhoods intersect $Y$ in a point which is not $x$ itself. Let
    \[
      Y' = \{x \in X \mid \text{$x$ is a limit point of $Y$} \}
    \]
  \end{enumerate}
\end{defn}
\trans{limit point}{gr{\"a}nspunkt}
\begin{example}
  Let $Y = [0,1) \cup \{2\} \subset \bbR$ with the standard topology on $\bbR$. Then $\Int Y = (0,1)$, $\bar{Y} = [0,1] \cup \{2\}$, $\dd Y = \{0,1,2\}$, and $Y' = [0,1]$.
\end{example}
\begin{thm}
  Let $X$ be a topological space and $Y \subset X$ a subset. Then
  \begin{enumerate}
    \item[(i)]  $\dd Y = X \setminus (\Int Y \cup \Int (X \setminus Y)) = \bar Y \cap \bar{X \setminus Y}$,
    \item[(ii)] $\bar{Y} = Y \cup \dd Y$, and
    \item[(iii)] $\bar{Y} = Y \cup Y'$.
  \end{enumerate}
\end{thm}
\begin{proof}
  The second equality in (i) is obtained directly from \eqref{int-versus-closure}, so it suffices to show the first equality. Taking complements, this becomes
  \[
    X \setminus \dd Y = \Int Y \cup \Int (X \setminus Y).
  \]
  Let $x \in X \setminus \dd Y$. This means that there is a neighbourhood $U$ of $x$ so that $x \in U \subset Y$ or $x \in U \subset X \setminus Y$. This means that either $x \in \Int Y$ or $x \in \Int (X \setminus Y)$.
  
  Suppose that either $x \in \Int Y$ or $x \in \Int (X \setminus Y)$. If $x \in \Int Y$ then there is an open set $U$ such that $x \in U \subset Y$, and so $U \cap Y^c = \emptyset$, so $x \notin \dd Y$. Similarly, if $x \in \Int(X \setminus Y)$, one gets an open neighbourhood $U$ of $x$ with $U \cap Y = \emptyset$, so once again $x \notin \dd Y$. This shows (i).
  
  From (i) and \eqref{int-versus-closure} it follows that
  \[
    Y \cup \dd Y = Y \cup (\bar Y \cap \bar{X \setminus Y}) = (Y \cup \bar Y) \cap (Y \cup \bar{X \setminus Y}) = \bar Y \cap X = \bar Y,
  \]
  which is (ii).
  
  Finally, we use (ii) and show that $Y \cup \dd Y = Y \cup Y'$. To see this, it suffices to show that $\dd Y \setminus Y = Y' \setminus Y$ as one can then take the union with $Y$ on both sides. So, let $x \in \dd Y \setminus Y$. Then any neighbourhood $U$ of $x$ intersects $Y$, and since $x \notin Y$, this necessarily means that $U$ intersects $Y$ in a point which is not $x$ itself, so $x \in Y'$, and since $x \notin Y$, we have $x \in Y' \setminus Y$.
  
  On the other hand, if $x \in Y' \setminus Y$, any neighborhood $U$ of $x$ will intersect $Y$; it will also intersect $X \setminus Y$, since $x$ belongs to that set. This implies that $x \in \dd Y$, and as before, $x \in \dd Y \setminus Y$. This completes the proof.
\end{proof}
\begin{example}
  We claim that $\bbQ$ is dense in $\bbR$, i.e. that $\bar \bbQ = \bbR$. By the above theorem, it suffices to show that $\dd \bbQ \cup \bbQ = \bbR$. To see this, let $x \in \bbR$ be arbitrary, and let $U$ be any neighbourhood of $x$. Now, an open set like $U$ is the union of a number of intervals, any interval contains (an infinite number of) both rational and irrational numbers, that is, $U \cap \bbQ \not= \emptyset$, and $U \cap (\bbR \setminus \bbQ) \not= \emptyset$. This is exactly the condition that $x \in \dd \bbQ$. Notice, not only did we show that $\dd \bbQ \cup \bbQ = \bbR$; we actually see that $\dd \bbQ = \bbR$.
\end{example}
\begin{prop}
  Let $\{X_i\}_{i \in I}$ be a family of topological spaces, and let $A_i \subset X_i$ be subsets of each of them. Then
  \[
    \prod_{i \in I} \bar {A_i} = \bar{\prod_{i \in I} A_i}.
  \]
\end{prop}
\begin{proof}
  Let $x = \{x_i\}_{i \in I} \in \prod_{i \in I} \bar {A_i}$. Let $U = \prod_{i \in I} U_i$ be any of the basis elements for the product topology $x \in U$, i.e. such that $x_i \in U_i$ and such that $U_i$ is open in $X_i$ for every $i \in I$. Since also $x_i \in \bar A_i$ for all $i$, we can choose a $y_i \in U_i \cap A_i$ for all $i$, so $y = \{y_i\}_{i \in I} \in U \cap \prod_{i\in I} A_i$. In other words, any neighbourhood $U$ of $x$ contains points from $\prod_{i\in I} A_i$ so $x \in \bar{\prod_{i\in I} A_i}$.
  
  For the converse, suppose that $x \in \bar{\prod_{i\in I} A_i}$. Let $V_i$ be any open set that contains $x_i$. Then by definition of the product topology, $\pi_i^{-1}(V_i)\subset X$ is a neighbourhood of $x$ and so contains a point $y \in \prod_{i\in I} A_i$. This says that $y_i \in A_i \cap V_i$ so that $x_i \in \bar{A_i}$ and $x \in \prod_{i \in I} \bar{A_i}$.
\end{proof}

\subsection{Separation axioms and Hausdorff spaces}
In $\bbR^n$ we have a very good idea of what it means for two points to be separated: if they're different, they're separated. We now ask ourselves the question if its possible to separate points in general topological spaces, using only the data of open sets. Consider for instance the topological space $X = \{a,b,c\}$ with the topology $\calT = \{ \emptyset, \{a\}, \{b,c\} , X\}$. From the viewpoint of topology, there is no way to tell the points $b$ and $c$ apart using only open sets; we consider the points inseparable.

Now for general topological spaces, it turns out that there are plenty of meaningful ways -- called separation axioms -- to then actually separate points, some of which lead to more interesting mathematics than others. In this section, we give the first few of these, including the immensely important Hausdorff axiom.
\begin{defn}
  A topological space $(X,\calT)$ is called
  \begin{enumerate}
    \item[(i)] $T_0$ if for every pair $x, y \in X$, $x \not= y$, there exists a neighbourhood of $x$ that does not contain $y$, \emph{or} there exists a neighbourhood of $y$ that does not contain $x$.
    \item[(ii)] $T_1$ if for every pair $x,y \in X$, $x \not= y$, $x$ has a neighborhood not containing $y$, \emph{and} $y$ has a neighbourhood not containing $x$.
    \item[(iii)] $T_2$ or \emph{Hausdorff}\index{Hausdorff} if for every pair $x, y \in X$, $x \not= y$, there exists neighbourhoods $U_x$ and $U_y$ of $x$ and $y$ respectively so that $U_x \cap U_y \not= \emptyset$.
  \end{enumerate}
\end{defn}
Note that clearly, a $T_2$-space is $T_1$, and a $T_1$-space is $T_0$.
\begin{prop}
  A topological space $X$ is $T_1$ if and only if $\{x \}$ is closed for all $x \in X$.
\end{prop}
\begin{proof}
  Suppose first that $\{x\}$ is closed for all $x \in X$, and let $x,y \in X$, $x \not= y$. Then $X \setminus \{x\}$ is a neighbourhood of $y$ that does not contain $x$, and $X \setminus \{y \}$ is a neighbourhood of $x$ not containing $y$, so $X$ is $T_1$.
  
  For the converse, suppose that $X$ is $T_1$, and let $x \in X$. Now every $y \in X$ has a neighbourhood $U_y$ that does not contain $x$, and so
  \[
    \bigcup_{y \not= x} U_y = X \setminus \{ x\}
  \]
  is open, and $\{x\}$ is closed.
\end{proof}
\begin{example}
  Let $X$ contain at least two points, and endow $X$ with the trivial topology. Then $X$ is not $T_0$ (or $T_1$ or $T_2$), since the only neighborhood of a point $x$ is $X$.
\end{example}
\begin{example}
  If $X$ has the discrete topology, then $X$ is Hausdorff (and $T_1$ and $T_2$).
\end{example}
\begin{example}
  Any poset $(X,\preceq)$ with the poset topology (Proposition~\ref{poset-top-def}) is $T_0$: let $x,y \in X$, $x \not= y$, and suppose that $x \preceq y$. Then $P_y = \{ z \in X \mid y \preceq z\}$ is a neighbourhood of $y$ which does not contain $x$.
\end{example}
\begin{example}
  Let $X = \{a,b,c\}$ with the topology
  \[
    \calT = \{ \emptyset, \{b\}, \{c\}, \{b,c\}, X \}.
  \]
  Then $X$ is $T_0$ but not $T_1$.
\end{example}
\begin{example}
  All metric spaces (with the metric topology) are Hausdorff (Exercise~1.\ref{metric-Hausdorff}).
\end{example}
