\section{Compactness and sequential compactness}
\label{compactness}
\subsection{Compactness}
\begin{defn}
  Let $(X,\calT)$ be a topological space.
  \begin{enumerate}
    \item[(i)] A collection $\calU \subset \calT$ of open sets of called an \wordexp{open~cover}{{\"o}ppen~{\"o}vert{\"a}ckning}{cover}{{\"o}vert{\"a}ckning} of $X$ if $X = \bigcup_{U \in \calU} U$.
    \item[(ii)] The space $X$ is called \word{compact}{kompakt} if \emph{every} open cover $\calU$ of $X$ has a finite subcover, meaning that one can find finitely many open sets $U_1, \dots, U_n \in \calU$ so that $X = \bigcup_{i=1}^n U_i$.
  \end{enumerate}
\end{defn}
\trans{cover}{{\"o}vert{\"a}ckning}
\trans{compact}{kompakt}
\begin{example}
  Every finite topological space is compact, since there are only finitely many open sets. Thus, given any open cover $\calU$, the open cover $\calU$ is itself a finite subcover.
\end{example}
\begin{example}
  \label{reals-not-compact}
  The real line $\bbR$ is not compact: consider the open cover $\calU$ consisting of the open sets $U_n = (-n,n)$, $n \in \bbN$. Clearly, it is impossible to choose finitely many such $U_n$ and still have something that covers all of $\bbR$.
\end{example}
\begin{example}
  The subspace $A = \{1/n \mid n \in \bbN\} \subset \bbR$ is not compact. One can see that $U_n = \{1/n\}$ is an open set in the subspace topology, so letting $\calU = \{U_n \mid n \in \bbN\}$, we get an open cover of $A$. Clearly, we can not find a finite subcover, since any finite subcover would cover only finitely many points of the infinite set $A$.
\end{example}
\begin{example}
  Let $X = A \cup \{0\}$, where $A$ is the set from the previous example. We claim that $X$ is compact. Let $\calU$ be an arbitrary open cover of $X$. Then there is an open set $U \in \calU$ so that $0 \in U$. By definition of the topology on $\bbR$, $U$ will contain the points $1/n$ for all large enough $n$, say all $n > N$ for some $N$. Since $\calU$ is an open cover, we can also find open sets $U_1, \dots, U_N \in \calU$ so that $1/k \in U_k$ for all $k = 1, \dots, N$. We now see that the collection $U, U_1, \dots, U_N$ together form a finite subcover of $X$.
\end{example}
\begin{example}
  The half-open interval $(0,1] \subset \bbR$ is not compact since the open cover $\calU$ consisting of open sets $U_n = (\tfrac{1}{n},1]$, $n \in \bbN$, does not have a finite subcover, by more or less the same argument as in Example~\ref{reals-not-compact}. Similarly, $(0,1)$ is not compact, since the sets $(\tfrac{1}{n},1-\tfrac{1}{n})$ form an open cover with no finite subcover.
\end{example}
\begin{example}
  \label{interval-compact}
  The closed interval $[0,1] \subset \bbR$ is compact. This is a special case of the Heine--Borel theorem, Theorem~\ref{heine-borel}, which we show below.
\end{example}
In the following theorems, we will collect a number of properties of compact sets that we will use over and over again.
\begin{thm}
  \label{closed-in-compact}
  A closed subspace of a compact space is compact.
\end{thm}
\begin{proof}
  Let $A \subset X$ be closed, and assume that $X$ is compact. To show that $A$ is compact, let $\calU = \{U_i\}_{i \in I}$ be an open cover of $A$. That is, every $U_i$ is open in $A$ in the subspace topology. By definition, we can find for every $i \in I$ open subsets $V_i$ of $X$ so that $U_i = A \cap V_i$. Since the $U_i$ cover $A$, it follows that the family $\calV = \{V_i\}_{i \in I} \cup \{A^c\}$ is an open cover of $X$; open because $A$ was assumed to be closed. Since $X$ is compact, there is a finite subcover $V_{i_1}, \dots, V_{i_n} \in \calV$ of $X$. Going, we see that $V_{i_1} \cap A, \dots, V_{i_n} \cap A \in \calU$ form a finite subcover of $A$, which is what we wanted to prove.
\end{proof}
\begin{thm}
  \label{compact-in-Hausdorff}
  A compact subspace of a Hausdorff space is closed.
\end{thm}
\begin{proof}
  Assume that $X$ is a Hausdorff space, and let $A \subset X$ be compact. We want to show that $A^c$ is open, and we will use the usual trick of showing that very point in $A^c$ has an open neighbourhood contained entirely in $A^c$, so that $A^c = \Int A^c$.
  
  Let $x \in A^c$ be a fixed point (and notice that if $A^c = \emptyset$, there is little to prove). For every point $y \in A$, we can find disjoint neighbourhoods $U_y$ and $V_y$ of $x$ and $y$ respectively, since $X$ is Hausdorff. Now the collection $\{A \cap V_y\}_{y \in A}$ is an open cover of $A$, and since $A$ is compact, we can choose finitely many $y_1, \dots, y_n$ so that $\{A \cap V_{y_i}\}_{i=1,\dots,n}$ is a finite subcover. In particular, $A \subset V_{y_1} \cup \dots \cup V_{y_n}$.
  
  Let $U^x = U_{y_1} \cap \dots \cap U_{y_n}$. Now $U^x$ is open by (T3), and $U^x \subset A^c$: if $z \in U^x$, then $z \in V_{y_i}^c$ for every $i = 1, \dots, n$, so $z \in (V_{y_1} \cup \dots \cup V_{y_n})^c \subset A^c$.
\end{proof}
\begin{thm}
  Let $X$ and $Y$ be topological spaces, assume that $X$ is compact, and let $f : X \to Y$ be a continuous map. Then the image $f(X) \subset Y$ is compact. If furthermore $Y$ is Hausdorff, and $f$ is a bijection, then $f$ is a homeomorphism.
\end{thm}
\begin{proof}
  Let $\calU = \{U_i \}_{i \in I}$ be an open cover of $f(X)$ and let us find a finite subcover. Define $V_i = f^{-1}(U_i)$ for every $i$. Then $\{V_i\}_{i \in I}$ is an open cover of $X$ which has a finite subcover $\{V_{i_1}, \dots, V_{i_n}\}$ since $X$ is compact. Now clearly, the corresponding collection $\{U_{i_1}, \dots, U_{i_n}\}$ is a finite subcover of $f(X)$.
  
  Assume now that $Y$ is Hausdorff and $f$ is bijective. We have to show that $f^{-1}$ is continuous, so let $U \subset X$ be open, and let us show that $f(U)$ is also open. To do so, note that $U^c$ is closed and thus compact by Theorem~\ref{closed-in-compact}. By the first part of the theorem, $f(U)^c = f(U^c)$ is also compact. By Theorem~\ref{compact-in-Hausdorff}, this means that $f(U)^c$ is closed, so $f(U)$ is open.
\end{proof}
\begin{cor}
  If $f: X \to Y$ is continuous, $X$ is compact, and $Y$ is Hausdorff, then $f : X \to f(X)$ is a homeomorphism.  
\end{cor}
\begin{proof}
  This follows from the theorem above since $f(X)$ is Hausdorff by 
\end{proof}
\begin{example}
  If we trust Example~\ref{interval-compact} which says that $[0,1]$ is compact, it follows that $S^1$ is compact, since the map $f:[0,1] \to S^1$ given by $f(x) = (\cos(2\pi x),\sin(2\pi x))$ is continuous and surjective.
\end{example}

