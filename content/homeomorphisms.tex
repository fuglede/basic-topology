\section{Homeomorphisms}
\label{homeomorphisms}
Often in mathematics, when talking about objects as certains things coming with certain structures, we want to be able to say when two objects are ``the same''. Consider for instance the two topological spaces $X = \{ 1 , 2 , 3 \}$ and $Y = \{ 4, 5, 6 \}$ with the topologies
\begin{align*}
  \calT_X &= \{ \emptyset, \{1 \}, \{ 2 \}, \{ 1, 2 \}, \{1,2,3\} \}, \\
  \calT_Y &= \{ \emptyset, \{4 \}, \{ 5 \}, \{ 4, 5 \}, \{4,5,6\} \}.
\end{align*}
These spaces are not particularly different for if we identify $1 \leftrightarrow 4$, $2 \leftrightarrow 5$, $3 \leftrightarrow 6$, we have no way to tell them apart. This notion of being the same is made precise in the definition of a ``homeomorphism'' below.

The well-educated mathematics student will have likely come across this general idea before: we consider two vector spaces the same if there is a linear isomorphism from one to the other, we consider algebraic objects such as groups and rings the same if they are isomorphic, and if all we know about two given sets is that they are in bijection, we may as well treat them as the same. This language of ``objects'' being ``the same'' is unified in the branch of mathematics called \word{category theory}{kategoriteori}, sometimes referred to as \emph{abstract nonsense}\index{abstract nonsense}. We will be discussing category theory in any detail in these notes, but it is useful to be aware of its existence.

\subsection{Homeomorphisms}
In the example above, we notice that crucial property of two topological spaces that ``are the same'' is that they are in bijection and have the same open sets. This leads to the following definition.
\begin{defn}
  A bijection $f : X \to Y$ between two topological spaces is called a \word{homeomorphism}{homeomorfi} if $f$ and its inverse $f^{-1}$ are continuous. In this case, we say that $X$ and $Y$ are \word{homeomorphic}{homeomorfa} and we write $X \simeq Y$.
\end{defn}
Equivalently, since a bijection $f$ always satisfies $f = (f^{-1})^{-1}$, one could define a homeomorphism to be a bijection such that $f(U)$ is open whenever $U$ is. Notice also that $\simeq$ satisfies the property of an equivalence relation.
\trans{homeomorphism}{homeomorfi}\trans{homeomorphic}{homeomorfa}
\begin{example}
  In the example in the beginning of this section, the bijection $f : X \to Y$ given by $f(1) = 4$, $f(2) = 5$, $f(3) = 6$ is a homeomorphism.
\end{example}
\begin{example}
  Let $f : (-1 , 1) \to \bbR$ be the bijective map
  \[
    f(x) = \tan \left( \frac{\pi x}{2} \right)
  \]
  whose inverse is $f^{-1}(x) = \tfrac{2}{\pi} \arctan x$. Then both $f$ and $f^{-1}$ are continuous so $(-1,1)$ and $\bbR$ are isomorphic.
\end{example}
From the above example we conclude that two spaces that we are otherwise familiar with and think of as different may turn out to be the same from the viewpoint of topology. Roughly, since we don't care about the scale of $(-1,1)$ but only its open sets, we are able to stretch it as much as we please, and end up with something like $\bbR$
\begin{badjoke}
  Let $A$ be a typical topologist. Then $A$ is not able to tell the difference between her coffee mug and her donut.
\end{badjoke}
\begin{proof}
  The surfaces of the coffee mug and the donut are homeomorphic. See \url{https://upload.wikimedia.org/wikipedia/commons/2/26/Mug_and_Torus_morph.gif}.
\end{proof}
\begin{example}
  Let $B^n := B(0,1)$ be the unit ball in $\bbR^n$. Then $B^n \simeq \bbR^n$. This can be seen because the map $f : B^n \to \bbR^n$ given by
  \[
    f(x) = \frac{x}{1-\Abs{x}}
  \]
  is a continuous bijection with inverse
  \[
    f^{-1}(x) = \frac{x}{1+\Abs{x}}.
  \]
\end{example}
We will often be interested in functions that would be homeomorphisms if we were allowed to shrink the codomain appropriately.
\begin{defn}
  Let $X$ and $Y$ be topological spaces. A function $f: X \to Y$ is called an \word{embedding}{?} if $f : X \to f(X)$ is a homeomorphism; here $f(X)$ has the subspace topology from $Y$.
\end{defn}
\begin{example}
  If $X$ is a topological space and $Y \subset X$ a subspace, then the inclusion $\iota : Y \to X$ given by $\iota(x) = x$ is an embedding.
\end{example}

\subsection{Topological invariants}
Above, we have talked about what it means for two topological spaces to be the same. Often, one will be interested in the converse question of telling two topological spaces apart. As such, we consider topological spaces different if they are non-homeomorphic; for instance, if $X = \{a,b\}$ then we obtain two different topological spaces by equipping it with the trivial and the discrete topology.

\begin{defn}
  Let \textbf{Top} denote the collection of \emph{all} topological spaces. A \emph{topological invariant}, sometimes called a \emph{topological property}, is a function $f$ defined on \textbf{Top} so that if $X \simeq Y$, then $f(X) = f(Y)$.
\end{defn}
The important thing to note is that if $f$ is a topological invariant and $f(X) \not= f(Y)$, then $X$ and $Y$ are not homeomorphic. Thus we are lucky enough, we can use topological invariants to tell topological spaces apart.
\begin{example}
  Let $f : \textbf{Top} \to \{ \text{yes}, \text{no} \}$ be the function given by answering the question ``is $X$ Hausdorff?'' That is
  \[
    f(X) = \begin{cases} \text{yes}, &\text{ if $X$ is Hausdorff,}\\ \text{no}, &\text{ if $X$ is not Hausdorff.} \end{cases}
  \]
  Then $f$ is a topological invariant: if $X \simeq Y$ and $X$ is Hausdorff, then so is $Y$. For this reason, the property of being Hausdorff is often called a topological property. Again, one can turn this around and say that if $X$ is Hausdorff but $Y$ is not, then $X$ and $Y$ are not homeomorphic. Similarly, being $T_0$ or $T_1$ are topological properties. As is being first-countable and any other property that is defined using only in terms of open sets.
\end{example}

We will encounter many other topological properties later on, one of the most important ones being the fundamental group, which is to be introduced in Section~\ref{homotopy}.

