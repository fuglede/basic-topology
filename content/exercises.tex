\section{Exercises}
The exercises are split into four sets, corresponding to the four exercise sessions held as part of the course. Many of the exercises are collected from previous iterations of the course, and these in turn may originate from \cite{Mun}. A few have been inspired by exercises in \cite{DM}.

\subsection{Set \#1}

\begin{enumerate}
  \item Define a relation on $\bbR$ by
    \[
      C=\{(x,y) \mid x - y \in \bbZ \}.
    \]
    Show that $C$ is an equivalence relation and describe the set of equivalence classes of $C$.
  \item Describe all possible topologies on the set $X = \{a,b,c\}$.
  \item Let $X$ be a set, and let $\calT_1$ and $\calT_2$ be two different topologies on $X$. When is the identity map $\id : X \to X$ given by $\id(x) = x$ a continuous map from $(X,\calT_1)$ to $(X,\calT_2)$.
  \item
		Show that the subspace topology $\calT_Y$ is the smallest (meaning coarsest) topology on $Y\subset X$ for which the inclusion $\iota:Y \rightarrow X$ is a continuous map.
	
	\item Let $Y\subset X$ be an open subset of a topological space $X$. Show that a set $U \subset Y$ is open in the subspace topology on $Y$ if and only if $U$ is open in $X$.
	
	\item \label{universal-inclusion} Show Lemma~\ref{universal-inclusion-lemma}.
	
  \item \begin{itemize}
		\item[($a$)] Describe the open sets in the poset topology on $\{a,b,c,d\}$ defined by the relations $a\preceq b\preceq c$ and $a\preceq d$.
		\item[($b$)] Describe the open sets in the poset topology on $(\bbR,\leq)$.
	\end{itemize}
	
  \item The Euclidean space $\bbR^2$ can be identified with the Cartesian product $\bbR \times \bbR$. Use Lemma~\ref{compare-bases} to show that the standard topology on $\bbR^2$ equals the product topology from $\bbR$ (were each $\bbR$ has the standard topology).
  

  \item \label{metric-Hausdorff} Show that metric spaces are always Hausdorff.
  
  \item Show that if $X$ is Hausdorff, then so is any subset $Y\subset X$ with the subspace topology.
  
  \item Show that the product of two Hausdorff spaces is Hausdorff.
  
  \item Show that a topological space $X$ is Hausdorff if and only if the diagonal
  \[
    \Delta = \{(x,x) \in X \times X \mid x \in X \} \subset X \times X
  \]
  is closed in the product topology on $X$.
  
  \item \label{metric-first-countable} Let $(X,d)$ be a metric space, and let
  \[
    \calB = \{ B_d(x,1/n) \mid x \in X, n \in \bbN \}.
  \]
  Show that $\calB$ is a basis for the metric topology on $X$.
  
  \item Let $(Y,\preceq)$ be a totally ordered set made into a topological space with the order topology.
  \begin{itemize}
    \item[($a$)] Show that for any two distinct points $x, y \in Y$ there are disjoint neighbourhoods, $U$ and $V$, of $x$ and $y$ respectively, so that $u < v$ for all $u \in U, v \in V$. Conclude that $Y$ is Hausdorff.
    \item[($b$)] Let $X$ be any topological space, and let $f,g:X\to Y$ be two continuous functions. Show that the set $\{x \mid f(x)\preceq g(x)\}$ is closed in $X$.
  \end{itemize}
  
  \item Let $(X_1,d_1)$ and $(X_2,d_2)$ be metric spaces. Define a metric on $X_1 \times X_2$ by
	\[
	  d((x_1,x_2),(y_1,y_2)) = \max(d_1(x_1,y_1),d_2(x_2,y_2)).
  \]
  Show that the metric topology on $X_1 \times X_2$ induced $d$ is the product topology, where $X_1$ and $X_2$ have the metric topologies from $d_1$ and $d_2$ respectively.

  
  \item Let $X,Y,Z$ be topological spaces and consider a function $F:X\times Y\rightarrow Z$. We say that $F$ is \emph{continuous in each variable} if for each $y_0\in Y$ the function $h:X\rightarrow Z$ defined by $h(x)=F(x,y_0)$ is continuous, \emph{and} if for each $x_0\in X$ the function $g:Y\rightarrow Z$ defined by $g(y) = F(x_0,y)$ is continuous. Show that if $F$ is continuous, then $F$ is continuous in each variable.
  
  \item \begin{itemize}
		\item[($a$)] A poset topology is $T_0$. When is it $T_1$?
		\item[($b$)] If $X$ is a $T_0$-space with finitely many elements. Then we can define a relation
		\[x\preceq y \Leftrightarrow y\in \bigcap_{U\subset X\text{ open}, \,x\in U} U.\]
		Show that $\preceq$ is a partial order. What is the poset topology on $(X,\preceq)$?
	\end{itemize}
\end{enumerate}

\subsection{Set \#2}
This section is not yet fixed and will likely change before the exercise session.
\begin{enumerate}
  \item Show that a topological space $X$ is connected if and only if the following condition holds: if $X = C \cup D$ where $C$ and $D$ are disjoint closed subsets of $X$, then either $C = \emptyset$ or $D = \emptyset$.
  \item \label{exercise-products-connected} Show Theorem~\ref{products-connected} in the case where $I$ is infinite. Inspiration can be found in \cite[Thm.~4.5]{Fje}.
\end{enumerate}
\subsection{Set \#3}

\subsection{Set \#4}
