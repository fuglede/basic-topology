\section{Separation and countability axioms}
\label{separation}
We encountered the first separation axioms in Section~\ref{separation-1}; in this section we introduce further notions of separations. We have already seen how the property of being $T_2$ allows many useful results, and in the same spirit we will see how having more fine-grained separation allows for further characterisation of topological spaces.
\subsection{Separation -- part 2}
\begin{defn}
  A topological space is called \word{regular}{regulj{\"a}rt} if for every closed set $F \subset X$ and any point $x \in X \setminus F$ there exist open sets $U_x, U_F \subset X$ so that $x \in U_x$, $F \subset U_F$, and $U_x \cap U_F = \emptyset$.
  
  A regular $T_1$-space is called $T_3$\index{$T_3$}.
\end{defn}
\trans{regular}{regulj{\"a}rt}
Notice that claerly, $T_3$-spaces are Hausdorff.
\begin{prop}
  Metric spaces are regular.
\end{prop}
\begin{proof}
  Let $(X,d)$ be a metric space, let $F \subset X$ be closed, and let $x \in X \setminus F$. Since $X \setminus F$ is open, there is an $\eps > 0$ so that $B_d(x,\eps) \subset X \setminus F$. Now let
  \[
    V = \bigcup_{y \in F} B(y,\eps/2).
  \]
  Then $V$ is open and $F \subset V$. We claim that $V \cap B(x,\eps/2) = \emptyset$ which completes the proof. Assume that $z \in V \cap B(x,\eps/2)$. Then there is a $y \in F$ with $d(z,y) < \eps/2$. Since also $d(x,z) < \eps/2$, the triangle inequality implies that
  \[
    d(x,y) \leq d(x,z) + d(y,z) < \eps,
  \]
  which is impossible by definition of $\eps$.
\end{proof}
\begin{example}
  Consider the set $\bbR_K$ from Example~\ref{K-topology}. Then $\bbR_K$ is Hausdorff since $\bbR$ is Hausdorff, and since the topology on $\bbR_K$ is finer than the standard topology. Now $K$ is closed in $\bbR_K$ by definition of the $K$-topology but it is impossible to separate $0$ and $K$ with disjoint open sets: assume that we could, and let $U$ and $V$ be the corresponding neighbourhoods of $0$ and $K$ respectively. Choose a basis element $B$ with $0 \in B \subset U$. Now $B$ must be of the form $(a,b) \setminus K$ since all intervals around $0$ contain elements from $K$. Now take $n$ so large that $1/n \in (a,b)$ and choose a basis element $B'$ with $1/n \in B \subset V$. Then $B$ must be an interval, and clearly this interval intersects $(a,b) \setminus K$, so $U$ and $V$ intersect.
\end{example}
\begin{defn}
  A topological space $X$ is called \word{normal}{normalt} if for all closed sets $F,G \subset X$ there are open sets $U_F, U_G$ with $F \subset U_F$, $G \subset U_G$ and $U_F \cap U_G = \emptyset$. A $T_1$-space which is normal is called $T_4$\index{$T_4$}.
\end{defn}
\trans{normal}{normalt}
\begin{prop}
  Metric spaces are normal.
\end{prop}
\begin{proof}
  Exercise.
\end{proof}
\begin{prop}
  Compact Hausdorff spaces are normal.
\end{prop}
\begin{proof}
  Exercise.
\end{proof}
\begin{example}
  Just as we saw above that a $T_2$-space need not be $T_3$, a $T_3$-space need not be $T_4$: an example of a $T_3$-space which is not $T_4$ is the so-called \emph{Sorgenfrey plane}\index{Sorgenfrey plane} $\bbR_l \times \bbR_l$, where $\bbR_l$ was defined in Example~\ref{lower-limit-topology}. For details, see~\cite[\S 31, Example 3]{Mun}.
\end{example}

\begin{thm}
  A subspace of a Hausdorff-space is Hausdorff, and a product of Hausdorff spaces is Hausdorff. A subspace of a $T_3$-space is $T_3$, and a product of $T_3$-spaces is $T_3$.
\end{thm}
\begin{proof}
  The first part of the Theorem was Exercise~1.\ref{subspace-Hausdorff}. Likewise, in Exercise~1.\ref{products-Hausdorff} it is claimed that the product of two Hausdorff spaces is Hausdorff. Let us include a proof of the general case here: Let $\{X_i\}_{i \in I}$ be Hausdorff spaces, and let $x, y \in \prod_{i \in I} X_i$, $x \not= y$. Now for some $i \in I$, $x_i \not= y_i$, so choose $U$ and $V$, neighbourhoods of $x_i$ and $y_i$ respectively with $U \cap V = \emptyset$. Then $\pi_i^{-1}(U)$ and $\pi_i^{-1}(V)$ are disjoint neighbourhoods of $x$ and $y$ respectively.

  Let $Y \subset X$ be a subset of a $T_3$-space $X$. By Exercise~1.\ref{subspace-Hausdorff}, $Y$ is Hausdorff so in particular $Y$ is $T_1$. Let $y \in Y$ be a point, and let $F \subset Y$ be closed in $Y$.
\end{proof}

