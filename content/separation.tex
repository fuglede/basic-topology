\section{Separation and countability axioms}
\label{separation}
We encountered the first separation axioms in Section~\ref{separation-1}; in this section we introduce further notions of separations. We have already seen how the property of being $T_2$ allows for many useful results, and in the same spirit we will see how having more fine-grained separation allows for further characterisation of topological spaces.
\subsection{Separation -- part 2}
\begin{defn}
  A topological space is called \word{regular}{regulj{\"a}rt} if for every closed set $F \subset X$ and any point $x \in X \setminus F$ there exist open sets $U_x, U_F \subset X$ so that $x \in U_x$, $F \subset U_F$, and $U_x \cap U_F = \emptyset$.
  
  A regular $T_1$-space is called $T_3$\index{$T_3$}.
\end{defn}
\trans{regular}{regulj{\"a}rt}
Notice that $T_3$-spaces are Hausdorff by Proposition~\ref{points-closed-in-t1}.
\begin{prop}
  Metric spaces are regular.
\end{prop}
\begin{proof}
  Let $(X,d)$ be a metric space, let $F \subset X$ be closed, and let $x \in X \setminus F$. Since $X \setminus F$ is open, there is an $\eps > 0$ so that $B_d(x,\eps) \subset X \setminus F$. Now let
  \[
    V = \bigcup_{y \in F} B(y,\eps/2).
  \]
  Then $V$ is open and $F \subset V$. We claim that $V \cap B(x,\eps/2) = \emptyset$ which completes the proof. Assume that $z \in V \cap B(x,\eps/2)$. Then there is a $y \in F$ with $d(z,y) < \eps/2$. Since also $d(x,z) < \eps/2$, the triangle inequality implies that
  \[
    d(x,y) \leq d(x,z) + d(y,z) < \eps,
  \]
  which is impossible by definition of $\eps$.
\end{proof}
\begin{example}
  Consider the set $\bbR_K$ from Example~\ref{K-topology}. Then $\bbR_K$ is Hausdorff since $\bbR$ is Hausdorff, and since the topology on $\bbR_K$ is finer than the standard topology. Now $K$ is closed in $\bbR_K$ by definition of the $K$-topology but it is impossible to separate $0$ and $K$ with disjoint open sets: assume that we could, and let $U$ and $V$ be the corresponding neighbourhoods of $0$ and $K$ respectively. Choose a basis element $B$ with $0 \in B \subset U$. Now $B$ must be of the form $(a,b) \setminus K$ since all intervals around $0$ contain elements from $K$. Now take $n$ so large that $1/n \in (a,b)$ and choose a basis element $B'$ with $1/n \in B \subset V$. Then $B$ must be an interval, and clearly this interval intersects $(a,b) \setminus K$, so $U$ and $V$ intersect.
\end{example}
\begin{defn}
  A topological space $X$ is called \word{normal}{normalt} if for all disjoint closed sets $F,G \subset X$ there are open sets $U_F, U_G$ with $F \subset U_F$, $G \subset U_G$ and $U_F \cap U_G = \emptyset$. A $T_1$-space which is normal is called $T_4$\index{$T_4$}.
\end{defn}
\trans{normal}{normalt}
\begin{prop}
  \label{metric-spaces-normal}
  Metric spaces are normal.
\end{prop}
\begin{proof}
  Exercise.
\end{proof}
\begin{prop}
  \label{compact-hausdorff-normal}
  Compact Hausdorff spaces are normal.
\end{prop}
\begin{proof}
  Exercise~\ref{compact-hausdorff-are-normal-exercise}.
\end{proof}
\begin{example}
  As before, Proposition~\ref{points-closed-in-t1} implies that $T_4$-spaces are $T_3$, but just as we saw above that a $T_2$-space need not be $T_3$, a $T_3$-space need not be $T_4$: an example of a $T_3$-space which is not $T_4$ is the so-called \emph{Sorgenfrey plane}\index{Sorgenfrey plane} $\bbR_l \times \bbR_l$, where $\bbR_l$ was defined in Example~\ref{lower-limit-topology}. For details, see~\cite[\S 31, Example 3]{Mun}.
\end{example}

\begin{lem}
  \label{separation-squeeze-lemma}
  Let $X$ be $T_1$. Then
  \begin{itemize}
    \item[(i)] $X$ is $T_3$ if and only if for each $x \in X$ and every neighbourhood $U$ of $x$, there is a neighbourhood $V$ of $x$ with $x \in \bar V \subset U$, and
    \item[(ii)] $X$ is $T_4$ if and only if for every closed set $F \subset X$ and every open $U \subset X$ with $F \subset U$ there is an open set $V \subset X$ with $F \subset V \subset \bar V \subset U$.
  \end{itemize}
\end{lem}
\begin{proof}
  Assume that $X$ is $T_3$, let $x \in X$, and let $U$ be a neighbourhood of $x$. Then $F = X \setminus U$ is closed and we can find open disjoint subsets $V,W \subset X$ so that $x \in V$ and $F \subset W$. We claim that $\bar V \cap F = \emptyset$ from which it follows that $\bar V \subset U$. If $y \in F$, then $W$ is a neighbourhood of $y$ which does not intersect $V$ so $y \notin \dd V$, and $y \notin V$ so $y \notin \bar V$.
  
  For the converse, let $x \in X$ and let $F \subset X$ be closed with $x \notin F$. Then $X \setminus F$ is a neighbourhood of $x$, so we can find a neighbourhood $V$ of $x$ with $x \in V \subset \bar V \subset X \setminus F$. That is, $X \setminus \bar V$ is open, contains $F$, and is disjoint from $V$.
  
  For the second part of the theorem, one uses the same argument with $x$ replaced by a general closed set.
\end{proof}

\begin{thm}
  A subspace of a Hausdorff-space is Hausdorff, and a product of Hausdorff spaces is Hausdorff. A subspace of a $T_3$-space is $T_3$, and a product of $T_3$-spaces is $T_3$.
\end{thm}
\begin{proof}
  The first part of the Theorem was Exercise~\ref{subspace-Hausdorff}.
  
  Likewise, in Exercise~\ref{products-Hausdorff} it is claimed that the product of two Hausdorff spaces is Hausdorff. Let us include a proof of the general case here: Let $\{X_i\}_{i \in I}$ be Hausdorff spaces, and let $x, y \in \prod_{i \in I} X_i$, $x \not= y$. Now for some $i \in I$, $x_i \not= y_i$, so choose $U$ and $V$, neighbourhoods of $x_i$ and $y_i$ respectively with $U \cap V = \emptyset$. Then $\pi_i^{-1}(U)$ and $\pi_i^{-1}(V)$ are disjoint neighbourhoods of $x$ and $y$ respectively.

  Let $Y \subset X$ be a subset of a $T_3$-space $X$. By the first part, $Y$ is Hausdorff so in particular $Y$ is $T_1$. Let $y \in Y$ be a point, and let $F \subset Y$ be closed in $Y$. Let $\bar{F}$ be the closure of $F$ in $X$. Then $\bar{F} \cap Y = F$ so $y \notin \bar{F}$. By the $T_3$-property for $X$, we get open sets $U_y$, $U_{\bar F}$ with $y \in U_y$, $\bar F \subset U_{\bar F}$ and $U_y \cap U_{\bar F} = \emptyset$. Now the sets $Y \cap U_y$ and $Y \cap U_{\bar F}$ do the job; they are open in $Y$, disjoint, $y \in Y \cap U_y$ and $F \subset Y \cap U_{\bar F}$.
  
  Let $\{X_i\}_{i \in I}$ be $T_3$-spaces and let $X = \prod_{i \in I} X_i$. As before, $X$ is $T_1$ since $X$ is $T_2$. Let $x = (x_i)_{i \in I} \in X$ be a point. We will use Lemma~\ref{separation-squeeze-lemma} to show that $X$ is $T_3$, so let $U$ be any neighbourhood of $x$. By definition of the product topology we can find open sets $U_i$ in $X_i$ so that $x \in \prod_{i \in I} U_i \subset U$. Since each $X_i$ is $T_3$, Lemma~\ref{separation-squeeze-lemma} provides us with neighbourhoods $V_i$ of $x_i$ so that $x_i \in V_i \subset \bar{V_i} \subset U_i$; if $U_i = X_i$ we simply take $V_i = X_i$. That is, $V_i = X_i$ for all but finitely many $i$ so that $V = \prod_{i \in I} V_i$ is open in $X$, and by Proposition~\ref{closures-of-products}, we have $\bar V = \prod_{i \in I} \bar{V_i}$. Altogether we see that $x \in V \subset \bar V \subset U$, so $X$ is $T_3$.
\end{proof}


\subsection{Second countability}
\begin{defn}
  A topological space $X$ is called \word{second-countable}{?} if the topology on $X$ has a countable basis.
\end{defn}
\trans{second-countable}{?}
Notice that a second-countable space is always first-countable.
\begin{example}
  Euclidean space $\bbR^n$ is second-countable (Exercise~\ref{second-countable-exercise}).
\end{example}
\begin{thm}
  Let $X$ be second-countable. Then
  \begin{itemize}
    \item[(i)] every open cover of $X$ has a countable subcover, and
    \item[(ii)] there is a countable dense subset of $X$.
  \end{itemize}
\end{thm}
A general space which has the property in (i) is called \emph{Lindel{\"o}f}\index{Lindel{\"o}f}, and a space with the property in (ii) is called \word{separable}{separabelt}. The theorem then says that a second-countable space is Lindel{\"o}f and separable.
\begin{proof}
  Let $\{B_n\}_{n \in \bbN}$ be a countable basis for the topology on $X$.
  
  Let $\calU$ be an open cover of $X$, and construct a countable cover as follows: for every $n \in \bbN$, we put $U_n = \emptyset$ if $B_n$ is not contained in any $U \in \calU$, and otherwise we let $U_n = U$ for some $U$ with $B_n \subset U$. We need to show that $\{U_n\}$ cover $X$. So, let $x \in X$. Then there is a $U \in \calU$ so that $x \in U$, and since $U$ is open, there is a basis element $B_n$ with $x \in B_n \subset U$. Now $x \in B_n \subset U_n$, so $x \in \bigcup_{n\in \bbN} U_n$, so the $\{U_n\}$ cover $X$.
  
  For the second part, choose $x_n \in B_n$ for every $n \in \bbN$. For each $x \in X \setminus \{x_n\}_{n \in \bbN}$, and for any neighbourhood $U$ of $x$, there is an $n$ with $x \in B_n \subset U$. this implies that $x \in \bar{\{x_n\}_{n \in \bbN}}$, and since $x$ was arbitrary, $\bar{\{x_n\}_{n \in \bbN}} = X$.
\end{proof}
\begin{example}
  \label{non-second-countable}
  Let $X$ be an uncountable set with the discrete topology. Then $\{ \{x\} \mid x \in X\}$ is an open cover of $X$ which has no countable subcover, so $X$ is not second-countable.
\end{example}
\begin{thm}
  A second-countable $T_3$-space is normal (and thus $T_4$).
\end{thm}
\begin{proof}
  Let $X$ be a $T_3$-space with a countable basis $\{B_n\}_{n \in \bbN}$, and let $F$ and $G$ be closed in $X$. Since $X$ is $T_3$, every $x \in F$ has a neighbourhood $U_x$ which is disjoint from $G$. By Lemma~\ref{separation-squeeze-lemma}, we can also find a neighbourhood $V_x$ of $x$ with $\bar{V_x} \subset U_x$, and finally we can find a basis element $B_n$ so that $x \in B_n \subset V_x$. Carrying out this procedure for every $x \in F$, we obtain a countable family of basis elements $\{B^F_k\}_{k \in \bbN}$ that covers $F$ and whose closures do not intersect $G$. Now, let $\tilde{U}^F = \bigcup_{k \in \bbN} B^F_k$.
  
  By doing the same for all points in $G$, we find a countable family $\{B^G_k\}_{k \in \bbN}$ that covers $G$ and such that the closure of each basis element does not intersect $F$, and we let $\tilde{U}^G = \bigcup_{k \in \bbN} B^G_k$. Now $\tilde{U}^F$ and $\tilde{U}^G$ are open and contain $F$ and $G$ respectively, but they need not be disjoint.
  
  What we do instead is essentially remove all the problematic points from $\tilde{U}^F$ and $\tilde{U}^G$ as follows: for every given $n \in \bbN$, define
  \[
    \hat{B}^F_n = B^F_n \setminus \bigcup_{k=1}^n \bar{B^G_k}, \quad \hat{B}^G_n = B^G_n \setminus \bigcup_{k=1}^n \bar{B^F_k}
  \]
  Then $\hat{B}^F_n$ and $\hat{B}^G_n$ are open for all $n$ since we remove from an open set something closed (and in general, such a difference can be written as the intersection of two open sets). Let
  \[
    U_F = \bigcup_{n \in \bbN} \hat{B}^F_n, \quad U_G = \bigcup_{n \in \bbN} \hat{B}^G_n.
  \]
  Then the sets $U_F$ and $U_G$ are open, and we claim that $F \subset U_F$, $G \subset U_G$, and $U_F \cap U_G = \emptyset$.
  
  If $x \in F$, then $x \in B^F_n$ for some $n$. Since none of the $\bar{B^G_k}$ intersect $F$, we know that $x$ does not belong to any of these, so it follows that $x \in \hat{B}^F_n \subset U_F$. Now $G \subset U_G$ by the same logic.
  
  To see that $U_F$ and $U_G$ are disjoint, let $x \in U_F \cap U_G$. Then $x \in \hat{B}^F_n \cap \hat{B}^G_m$ for some $n$ and $m$. We see that this is impossible by definition of $\hat{B}^F_n$ and $\hat{B}^G_m$ by considering the two cases $n \leq m$ and $m \leq n$.
\end{proof}

\subsection{Urysohn's lemma}
By definition, a $T_4$-space is a space where disjoint closed sets can be separated by disjoint open sets containing the closed sets. In this section, we mention Urysohn's lemma, which says that disjoint closed sets can be separated by continuous functions, in a very concrete sense.
\begin{lem}[Urysohn's lemma]
  \index{Urysohn's lemma}
  Let $X$ be a $T_4$-space, let $F$ and $G$ be closed disjoint subsets, and let $a, b \in \bbR$ be real numbers with $a \leq b$. Then there is a continuous function $f : X \to [a,b]$ so that $f(F) = \{a\}$, $f(G) = \{b\}$.
\end{lem}
The proof is rather involved and unlike all other results that we have encountered so far, it is not sufficient to simply juggle definitions. Instead of giving a proof, which can be found in \cite[\S 33]{Mun}, we will provide an example of its power in the next section.

Let us end this section with a different application of Urysohn's lemma. Notice that so far, most of the concrete examples of topological spaces that we have considered have all been metric spaces. Likewise we know that any given set can be given both a topology and a metric, so there is a natural question: given a topological space $(X,\calT)$, is there a metric $d$ on $X$ so that $\calT$ is the metric topology? If so, we say that $X$ is \word{metrisable}{metriserbart}.
\begin{example}
  All metric spaces are metrisable.
\end{example}
\begin{example}
  All products of metrisable spaces are metrisable by Exercise~\ref{product-metric}.
\end{example}
\begin{example}
  Any discrete topological space is metrisable by Example~\ref{discrete-metric}.
\end{example}
If a topological space $X$ is metrisable, it has all the topological properties that general metric spaces have. For instance, all metrisable spaces are normal by Proposition~\ref{metric-spaces-normal}. Therefore a space which is not normal is also not metrisable, and such spaces exist.

Now Urysohn's metrisation theorem provide a sufficient condition for a topological space to be metrisable. A proof can be found in \cite[\S 34]{Mun}.

\begin{thm}[Urysohn's metrisation theorem]
  \index{Urysohn's metrisation theorem}
  All second-countable $T_3$-spaces are metrisable.
\end{thm}

Notice that the converse is not true: an uncountable metric space with the discrete metric (Example~\ref{discrete-metric}) is not second-countable (Example~\ref{non-second-countable}).
